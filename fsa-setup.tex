\section{Experimental Setup}\label{sec:fsa-setup}

Each experiment compares several file systems: \betrfs, \btrfs, \ext, \ftwofs,
\xfs, and \zfs.  We use the versions of \xfs, \btrfs, \ext and \ftwofs that are
part of the \linuxver kernel, and \zfs 0.6.5-234\_ge0ab3ab, downloaded from the
zfsonlinux repository on \url{www.github.com}.  We used BetrFS 0.3 in the
experiments\footnote{Available at \url{github.com/oscarlab/betrfs}}.  We use
default recommended file system settings unless otherwise noted.  Lazy inode
table and journal initialization are turned off on \ext, pushing more work onto
file system creation time and reducing experimental noise. 

All experimental results are collected on a Dell Optiplex 790 with a 4-core
3.40 GHz Intel Core i7 CPU, 4 GB RAM, a 500 GB, 7200 RPM ATA Seagate Barracuda
ST500DM002 disk with a \SI{4096}{\byte} block size, and a 240 GB Sandisk
Extreme Pro---both disks used SATA 3.0.  Each file system's block size is set
to \SI{4096}{\byte}.  Unless otherwise noted, all experiments are cold-cache.

The system runs 64-bit Ubuntu 13.10 server with Linux kernel version
\linuxver{} on a bootable USB stick.  All HDD tests are performed on two 20GiB
partitions located at the outermost region of the drive.  For the SSD tests, we
additionally partition the remainder of the drive and fill it with random data,
although we have preliminary data that indicates this does not affect
performance.
