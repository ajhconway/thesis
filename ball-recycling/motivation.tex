% !TEX root =  paper.tex

\section{Ball Recycling and Insertion/Update Buffers}\label{sec:motivation}

Ball recycling models insertion and update buffers, which are widely used in
modern databases~\cite{Azure16, IBM17, Xiang12, Informix, Callaghan11, NuDB16,
Oracle17, SAP17, Vertica17,CanimLaMi10,BenderFaJo12}. These implementations are
discussed in more detail in \Cref{sec:experiments}.

For a key-value store, such as a database, an \defn{insertion buffer} is a
cache of recently inserted items.  When the insertion buffer fills, the
database selects a disk block and all the cached items going to that block are
evicted in bulk.  If $k$ elements are flushed in bulk, then there is a speedup
of $k$, compared to writing the elements to the destination block as soon as
they arrive. After evicting $k$ items, there is room for $k$ new elements in
the buffer.  An \defn{update buffer} caches changes to existing key-value pairs
but is otherwise like an insertion buffer.  Although these types of buffers
seem quite similar, we show that they have important differences in how they
are modeled as a ball-recycling game.

The mapping to ball recycling is direct: disk blocks are bins and elements in
the insertion/update buffer are balls.  The probability distribution $\p$ is
based on the distribution of items inserted  or updated.  Evicting all the
items going to a disk block corresponds to emptying the bin associated with
that disk block.  After an eviction of $k$ items, we have room for $k$ new
insertions/updates, i.e., we have $k$ new balls to throw.  The policy for
selecting the target disk block of an eviction corresponds to the policy for
selecting a bin to recycle, and the speedup induced by an eviction policy is
its recycling rate.  

For \btrees{}, insertion buffers and update buffers differ in an important way:
updates do not change the structure of leaves of a \btree{}. In contrast,
insertions can change the range of keys associated with a leaf (due to leaf
splits), which yields the following result:

\begin{lemma}\label{lem:uniform-leaves}
	If $N$ keys are inserted into a \btree{} i.i.d.\ according to some key
	distribution $\q$, then provided $B = \Omega(\log{N})$, the maximum
	probability that a leaf has of receiving the next insertion is $O(B/N)$
	with probability 1.  Thus the corresponding recycling game is
	asymptotically \defn{almost uniform}: no bin has probability more than a
	constant multiple of $\frac{1}{n}$.
\end{lemma}

We prove the uniformity bound as follows.  Let $F(\kappa)$ be the cumulative
density function (CDF) of $\q$, which is the probability that an item sampled
from $\q$ is less than $\kappa$.  If $\kappa$ is distributed according to $\q$,
then $F(\kappa)$ will be uniformly distributed on $[0,1]$.

If $n$ points are sampled from $[0,1]$ uniformly and sorted so that $x_1
\leq x_2 \leq \cdots \leq x_n$, then $\max{x_{i+B}-x_i}$ is known as the
\defn{maximal $B$-spacing}. It follows that:

\begin{lemma}\label{lem:leaf-b-spacing}
	Having inserted $n$ keys into a \btree{} i.i.d.\ according to a distribution
	$\q$, the maximum probability that any leaf has of receiving the next
	insertion is less than the maximum $B$-spacing of the CDFs of those points.
\end{lemma}

It is known that:

\begin{lemma}[\cite{DeheuvelsDe84}]
	If $B = \Theta(\log{n})$, then the maximum $B$-spacing of $n$ points
	distributed uniformly on the unit interval is $\Theta(B/n)$ with
	probability 1.
\end{lemma}

For $B = \omega(\log{n})$, we can subdivide $B$ into intervals of $\log{n}$
points, each of which will satisfy the lemma. Adding together the resulting
bounds, we have that the maximal $B$-spacing of $n$ points is $O(B/n)$ with
probability 1. Together with \Cref{lem:leaf-b-spacing}, this implies
\Cref{lem:uniform-leaves}.

We also note that an (almost-uniform) ball-recycling game is an imperfect model
for an insertion buffer, because the ball-recycling game has a fixed number of
bins, whereas in the insertion buffer, the number of disk blocks will increase.
Furthermore, insertions may not be independently distributed.

Finally, the implementation of these strategies is a point of departure between
insertion/update buffers and ball recycling.  In ball recycling, it is obvious
which bin each ball is in.  In insertion/update buffers for a \btree{},
elements have a key, but we don't necessarily know what the buckets are, since
the mapping from keys to buckets depends on the pivots used to define the
\btree{} leaves.  \FB needs to know what the buckets are, whereas \RB and \GG
do not.  For \RB this is because the key of the randomly selected item can be
used to fetch its target \btree{} leaf, after which we know the max and min
keys in that leaf, and \GG can be implemented by remembering the upper bound of
the last leaf to which we flushed and then flushing the item with the successor
of that key, along with all the other keys going to that leaf.  None of these
strategies require knowledge of $\p$. 

Our results on general $\p$ have an interesting implication for \betrees{},
which are known to be asymptotically optimal for insertions, in the worst case.
\betrees{} can also handle updates by propagating messages to the leaves.  For
some update distributions, flushing according to \RB can achieve an update rate
that is $B^\varepsilon$ faster than \FB.  We expect to try \RB flushing in our
\betree{}-based file system,
B$\varepsilon$trFS~\cite{JannenYuZh15a,YuanZhJa16,YuanZhJa17,ConwayBa17,JannenYuZh15,esmet2012tokufs}.


%%% Local Variables:
%%% mode: latex
%%% TeX-master: "paper.tex"
%%% End:
