% !TEX root =  paper.tex

\secput{related}{Related Work}

Balls-and-bins games are one of the most studied models in all of computer
science, so it would be impossible to do justice to the entire literature here.
Rather, we focus on prior work on \defn{dynamic} balls-and-bins games.  In
dynamic balls-and-bins games, balls are added and removed from bins according
to some rules, and the goal is to understand the long-term behavior of the
system.  Thus, ball-recycling games are instances of this general class,
although previously studied dynamic balls-and-bins games are quite different,
and are typically used to study load-balancing problems, rather than
throughput.

Previously studied models differ from ball-recycling games in several
ways: 
\begin{itemize}
\item The process for removing balls is assumed to be random, e.g. a
  random ball is removed, or a random bin is selected for emptying.
  Prior research has assumed these events are determined by some
  external process, so they have not studied algorithms for
  controlling this process.  As a result, although the theory of
  Markov chains has played a major role in the study of balls-and-bins
  games, the theory of Markov decision processes, which we use
  extensively, has not shown up at all.
\item The balls are thrown uniformly randomly.  This is a natural
  assumption for hashing and randomized load-balancing problems, but
  is not appropriate when studying updates to existing keys in a
  database.
\item The objective is to analyze the occupancy of the fullest bin or,
  in some models, to analyze the amount of time that balls wait in a
  queue.  Our objective is to analyze the number of balls recycled in
  each time step.
\item The number of balls in the system is not fixed.  Prior models
  were used for load balancing, in which balls correspond to tasks and
  bins resources, so it makes sense to model new balls entering the
  system asynchronously.  In our setting, the balls are the
  resource (i.e. they correspond to slots in a buffer), so
  they are fixed.
\item The study of dynamic balls-and-bins games was introduced
  simultaneously with the study of the power of multiple
  choices~\cite{AzarBrKa94}, and most past work on dynamic
  balls-and-bins games has been in the same model.  In our model,
  balls have only a single choice (i.e. their on-disk location),
  although it may be possible to extend our work to systems in which
  each ball has multiple choices (e.g. for an insertion buffer for an
  on-disk cuckoo hash table).
\end{itemize}

Azar, et al.~\cite{AzarBrKa94} introduced both the power of two
choices and dynamic balls and bins games.  They showed that if, at
each time step, a random ball is rethrown with $d$ uniformly random
choices, then, in $n^3$ time steps, the fullest bin has $\ln \ln n /
\ln d + O(1)$ balls w.h.p.

In his dissertation~\cite{Mitzenmacher96}, Mitzenmacher studied several dynamic
load-balancing problems.  This included several variants on the supermarket
model, in which balls arrive according to a Poisson process and enqueue
themselves in the shortest of $d$ queues that they select uniformly randomly
from $n$ queues.  Mitzenmacher showed that $d>1$ exponentially reduced the
average time a ball spent in a queue.  He also studied a variant in which, at
each time step, one ball was removed from one queue and was immediately
re-enqueued according to the above procedure.  Adler, et al.~\cite{AdlerBeSc98}
studied a variant of the supermarket model in which balls arrived in batches of
size $m$ and chose their queues in parallel, and showed that the average
waiting time remains $O(\ln \ln n)$ as long as $m$ is sufficiently smaller than
$n$.

Cole, et al.~\cite{ColeFrMa98} studied a model in which balls are recycled
one-at-a-time according to a recycling plan chosen in advance.  In their model,
% density-dependent jump Markov processes.
there are an infinite number of labeled balls, and the adversary specifies in
advance two ball IDs for each time step: the first ID specifies a ball to be
removed from the system, and the second ID specifies a ball to be inserted.
Thus the number of balls currently in the system is always $n$.  The first time
a ball is inserted, it chooses $d$ bins uniformly at random and picks the least
loaded. From then on, whenever that ball reenters the system, it always goes
into that bin.  In their model, the adversary cannot examine the state of the
current system when deciding which ball to recycle, as in our model.  Given
this restriction, they show that the fullest bin has roughly $\ln \ln n / \ln
d$ balls.  V\"{o}cking~\cite{Voecking03} showed the shocking result that, by
choosing bins non-uniformly and breaking ties asymmetrically, the max load
could be reduced to $\ln \ln n / d\ln\phi_d+O(1)$ w.h.p.

Cole, et al.~\cite{ColeMaMe98} extended their results to routing through a
network: an adversary specified in advance the start time, end time, source,
and destination of flows, and the system used a power of two choices variant of
Valiant's randomized routing paradigm~\cite{ValiantBr81} to limit congestion to
$O(\ln \ln n)$ w.h.p.

Czumaj and Stemann~\cite{CzumajSt97} study a load balancing problem in which,
at each time step, a random ball is removed from the system and a new ball is
thrown using $d$ choices.  After the new ball is inserted, the $d$ bins
examined during its insertion are rebalanced.  Surprisingly, the max load is
still $O(\ln \ln n)$.


%%% Local Variables:
%%% mode: latex
%%% TeX-master: "paper.tex"
%%% End:
