% !TEX root =  paper.tex

\secput{uniform}{The Uniform Case}

The results of \Cref{sec:nonuniform} hold for any distribution of the balls
into the bins. In this section we consider the special case where they are
uniformly distributed, which models insertion buffers as discussed in
\Cref{sec:motivation}.  We then show that \GG and \FB are optimal, up to
lower-order terms, in this setting, whereas \RB is at least $1/2$- and at most
$(1-\epsilon)$-optimal, for some constant $\epsilon>0$.

For a ball-recycling game with uniformly distributed balls,
\Cref{lem:upperbound} implies:
\begin{corollary}\label{cor:uniformupperbound}
	Consider a ball-recycling game with $m$ balls, $n$ bins and uniform
	distribution $\uni$. For any recycling strategy $A$, 
	\[\mathcal{R}^A \leq \frac{2m+n-1}{n} < 2\frac{m}{n} + 1.\]
\end{corollary}

The average number of balls in a bin is $m/n$, so \Cref{cor:uniformupperbound}
suggests that any ``reasonable'' strategy will be at least $1/2$-optimal in the
uniform case.

We now show that \GG and \FB are within an additive constant of optimal on strictly
uniform distributions.

\begin{lemma}\label{lem:unilowerbound}
	\GG and \FB each recycle at least $2m/(n+1)$ balls per round in
	expectation.
\end{lemma}

\begin{proof}
	Let $S$ be the random variable denoting the number of balls thrown in a
	given round with \GG. \GG will recycle the bins in order starting from the
	next one and cycling around. Therefore, we can consider the collection of
	bins to be a queue. After throwing the balls, the average place in the
	queue in which a ball lands is the $\left[(n-1)/s\right]$th bin, due to
	uniformity. Each ball thrown will therefore sit for an average of at most
	$(n-1)/2$ rounds before it is thrown again. Therefore,
	$m-\E{S}\leq\E{S}(n-1)/2$, and we have the result after solving for
	$\E{S}$.

	Let $T$ be the random variable denoting the number of balls thrown in a
	given round with \FB. If after removing the balls in the \FB, we list the
	bins in order of fullness, we can again think of the bins as a sort of
	queue.  When we throw the balls, the average place in the queue which a
	ball lands is the $\left[(n-1)/2\right]$th bin as above, due to uniformity.
	Now, we reorder the bins back into fullness order.  During the reordering
	more balls are moved up the queue than down, thus each ball thrown into the
	system will sit for an average of less than $(n-1)/2$ rounds before it is
	thrown again. Therefore, as above, $m-\E{S}\leq\E{S}(n-1)/2$, and we are
	done.
\end{proof}

\Cref{cor:uniformupperbound} and \Cref{lem:unilowerbound}
together prove \Cref{thm:fullestbin}. Despite these strong
performance bounds, recall that \FB can perform arbitrarily badly on
non-uniform $\p$. \RB on the other hand is always $\Theta(1)$-optimal.

\subsection{\RB in the Uniform Case}

However, \RB{} does not achieve this level of optimality on uniform
distributions. In this section we will show in \Cref{thm:rbuniform} that \RB
recycles at most $1+(2-\epsilon)m/n$ balls per round in expectation, for some
$\epsilon > 0$.  The upper bound is given in \Cref{lem:rbuni-upper} and
\Cref{cor:rbuni-upper-constants}, and the lower bound is given in
\Cref{lem:rbuni-lower}.

We begin with the following lemma:

\begin{lemma}\label{lem:pair-flow}
	Let $\chi^\textrm{RB}$ be the stationary distribution relative to \RB,
	$R^\textrm{RB}(X)$ the random variable of how many balls \RB recycles from
	ball configuration $X$, and $\mathcal{R}^\textrm{RB} =
	\E{R^\textrm{RB}\left(\chi^\textrm{RB}\right)}$ the expected recycle rate
	of \RB. Then,
	\[ \frac{\E{R^\textrm{RB}\left(\chi^\textrm{RB}\right)^2}}{\mathcal{R}^\textrm{RB}} = \frac{2m+n-1}{n+1} \leq 1 + \frac{2m}{n}. \]
\end{lemma}

\begin{proof}
	Consider the random variable of the number of distinct unordered pairs of
	balls which are in the same bin in $\chi^\textrm{RB}$. In expectation, a
	round of \RB eliminates 
	\[\binom{\mathcal{R}^\textrm{RB}}{2}\]
	and creates
	\[\sum_{k=0}^{\mathcal{R} - 1} \frac{m - \mathcal{R}^\textrm{RB} + k}{n}\]
	such pairs. In the stationary distribution, these must be equal, so
	\begin{multline*}
		\frac{\E{R^\textrm{RB}\left(\chi^\textrm{RB}\right)^2}}{2} - \frac{\mathcal{R}^\textrm{RB}}{2} \\
		= \frac{(2m-1)\mathcal{R}^\textrm{RB}}{2n} - \frac{\E{R^\textrm{RB}\left(\chi^\textrm{RB}\right)^2}}{2n}.
	\end{multline*}
	After rearranging we have the result.
\end{proof}

\begin{lemma}\label{lem:rbuni-upper}
	There exists a constant $\alpha > 0$ such that \RB is at most
	$(1-\alpha)$-optimal.
\end{lemma}

\begin{proof}
	Let $\chi^\textrm{RB}$ be the stationary distribution relative to \RB,
	$R^\textrm{RB}(X)$ the random variable of how many balls \RB recycles from
	ball configuration $X$, and $\mathcal{R}^\textrm{RB} =
	\E{R^\textrm{RB}\left(\chi^\textrm{RB}\right)}$ the expected recycle rate
	of \RB.  We will prove the result by contradiction, so assume that for all 
	constant $\epsilon > 0$
	\[\mathcal{R}^\textrm{RB} \geq 1 + \frac{(2-\epsilon)m}{n}.\]

	Let $c\in(1,2)$ be a constant to be determined later. We say a bin is
	\defn{light} if it contains at most $cm/n$ balls. Let $L$ be the random
	variable of the number of balls in light bins in the stationary
	distribution. Then the probability $q_L$ that \RB recycles a light bin in
	the stationary distribution is $\E{L}/m$. We proceed by cases.

{\bf Case 1.} Suppose $\E{L} \geq \delta m$ for some constant $\delta > 0$. Then $q_L \geq \delta$ and for $c \leq 2 - 2\epsilon$ and $\epsilon < 1/2$,
\begin{align*}
	\Var{R^\textrm{RB}\left(\chi^\textrm{RB}\right)} 
	&= \E{\left(R^\textrm{RB}\left(\chi^\textrm{RB}\right) - \mathcal{R}^\textrm{RB}\right)^2}  \\
	&\ge q_L \left(1+\frac{(2-\epsilon)m}{n}-\frac{cm}{n}\right)^2 \\
%	&\ge \delta(2-\epsilon-c)^2 \frac{m^2}{n^2} + \\ &2\delta(2-\epsilon-c)\frac{m}{n}+\delta \\
	&\ge \frac{\epsilon^2\delta}{4} \left(\frac{4m^2}{n^2}+\frac{4m}{n}+1\right) \\
	&\ge \frac{\epsilon^2 \delta}{4} \left(\mathcal{R}^\textrm{RB}\right)^2. \label{eqn:varFl}
\end{align*}

Thus by the definition of variance, we have
\begin{equation*}
	\E{R^\textrm{RB}\left(\chi^\textrm{RB}\right)^2} \ge \left(1+\frac{\epsilon^2\delta}{4}\right)\left(\mathcal{R}^\textrm{RB}\right)^2. \label{eqn:expF2l}
\end{equation*}
	Now by \Cref{lem:pair-flow},
\begin{equation*}
	\mathcal{R}^\textrm{RB} \le \left(1+\frac{\epsilon^2\delta}{4}\right)^{-1}\left(1+\frac{2m}{n}\right).
\end{equation*}
Since $\epsilon, \delta$ are constants greater than $0$, we have our contradiction for
	the first case. 

\medskip 

{\bf Case 2.} Otherwise, $\E{L} < \delta m$. Since $L \in [0, m]$, $\E{L^2} <
	\delta m^2$. \Cref{lem:pair-flow} implies $\E{R^\textrm{RB}\left(\chi^\textrm{RB}\right)^2} \le (1+2m/n)^2$. Together H\"{o}lder's inequality we have
\begin{align}
	\E{LR^\textrm{RB}\left(\chi^\textrm{RB}\right)}
	&\le \left(\E{L^2}\E{R^\textrm{RB}\left(\chi^\textrm{RB}\right)^2}\right)^{1/2} \nonumber \\
	&< \left(\delta m^2 \left(1+\frac{2m}{n}\right)^2\right)^{1/2} \nonumber\\
	&= \sqrt{\delta} m\left(1 + \frac{2m}{n}\right) \label{eqn:expLFu}
\end{align}

	Let $Y$ be the random variable of the number of balls in the stationary
	distribution which start in a light bins, but end up begin among the first
	$1 + cm/n$ balls in a heavy bin after an application of \RB.  Let $\Phi$
	be the random variable of the number of distinct unordered pairs of balls
	that are in the same light bin in the stationary distribution. Applying \RB
	in expectation creates at most
	\begin{multline*}
          \E{\sum_{k=0}^{\mathcal{R}^\textrm{RB}-1}\frac{L + k}{n}} \\
		  = \E{\frac{2LR^\textrm{RB}\left(\chi^\textrm{RB}\right) + R^\textrm{RB}\left(\chi^\textrm{RB}\right)^2 - R^\textrm{RB}\left(\chi^\textrm{RB}\right)}{2n}}
	\end{multline*}
	such pairs, and eliminates at least
	\[ \E{\frac{Y}{1+\frac{cm}{n}}\binom{1+\frac{cm}{n}}{2}}.\]
	In the stationary distribution these quantities must be equal, so
	rearranging together with Equation~\eqref{eqn:expLFu}, we have
	
\begin{align*}
	\E{Y} 
	&\le \frac{2\E{LR^\textrm{RB}\left(\chi^\textrm{RB}\right)} + \E{R^\textrm{RB}\left(\chi^\textrm{RB}\right)^2} - \mathcal{R}^\textrm{RB}}{cm} \nonumber \\
	&< \frac{2\sqrt{\delta}}{c}\left(1+\frac{2m}{n}\right) + \frac{\E{R^\textrm{RB}\left(\chi^\textrm{RB}\right)^2} - \mathcal{R}^\textrm{RB}}{cm} \\
	&<\left(1+\frac{2m}{n}\right)\left(\frac{2\sqrt{\delta}}{c}+\frac{2}{cn}\right),  
\end{align*}
where we have used \Cref{lem:pair-flow} for the last inequality.

We now compute the effect on $\E{L}$ of applying \RB to the stationary
distribution.  By Markov's inequality, there must be more than $(1-1/c)n$ light
bins, and so the probability that a ball is thrown into a light bin is more
than $1-1/c$. Therefore, at least $(1-1/c)\mathcal{R}^\textrm{RB}$ balls land
in light bins in expectation. We expect at most $\E{Y}$ balls to be in light
bins which turn into heavy bins. Finally, we recycle at most $cm/n$ balls from
a light bin $\E{L}/m$ of the time. Since the net change to $L$ must be $0$ in
expectation,
\[\left(1-\frac{1}{c}\right)\mathcal{R}^\textrm{RB} < \frac{c}{n}\E{L} + \E{Y}.\]

However, this is a contradiction. Indeed, the LHS is at least 
$$
\left(1-\frac{1}{c}\right)\left(1+\frac{(2-\epsilon)m}{n}\right),
$$
but the RHS is less than
$$
\delta c \frac{m}{n} + \left(1+\frac{2m}{n}\right)\left(\frac{2\sqrt{\delta}}{c}+\frac{2}{cn}\right).
$$
Thus, if we pick a sufficiently small $\delta > 0$, $\epsilon = 0.01$, $c=1.98$
and $n\geq 3$, we have a contradiction. For $n \le 2$, the contradiction follows
immediately from \Cref{lem:pair-flow}.
\end{proof}

\begin{corollary}\label{cor:rbuni-upper-constants}
	Setting
	\begin{equation*}
		(\epsilon, c, \delta) = (0.001, 1.456, 0.042)
	\end{equation*}
	in the proof of \Cref{thm:rbuniform}, we obtain 
	\begin{equation*}
		\mathcal{R}^\textrm{RB} < 1 + 1.994\frac{m}{n}.
	\end{equation*}
\end{corollary}

\begin{lemma}\label{lem:rbuni-lower} For all $c > 0$, there exists a
  $c'$ such that if $m \ge c' n\log n$, the uniform random ball policy
  has expected recycle rate at least   
$$
 	\left(1+\frac{1}{6^4} - c\right)\frac{m}{n}.
$$
\end{lemma}

\begin{proof}
	Let $X_{t, k}$ be the random variable denoting the number of balls in the
	$k$th bin at the beginning of the $t$th round. Because of symmetry,
	$X_{t, k}$ follows the same distribution as $X_{t, \ell}$ for any $k \ne
	\ell$.  For simplicity, we let $X_t$ be a random variable that follows the
	same distribution as $X_{t, k}$ for all $k$. 

	We pick $t$ to be sufficiently large so that the system enters its
	stationary state after $t$ rounds. Thus, $X_t$ and $X_{t'}$ follows the
	same distribution for any $t' > t$.

	Let $Y_t$ be the random variable denoting the number of balls recycled in
	the $t$th round. By definition, we have
	$$ \E{Y_t} = \sum_{1 \le k \le n}\frac{\E{X_{t, k}^2}}{m} =
	\frac{n}{m} \left( \E{X_t}^2 + \Var{X_t} \right).  $$
	Note that $\E{X_t} = m/n$, and so $\E{Y_t} \ge m/n$. 

	To show $\E{Y_t}$ deviates from $m/n$, we derive a lower bound for
	$\Var{X_t}$.

	If $\Prob{X_t \le (1-\epsilon)m/n} \ge \delta$, then $\E{Y_t} \ge
	(1+\epsilon^2\delta)m/n$.

	Otherwise $\Prob{X_t > (1-\epsilon)m/n } > 1-\delta$. We will show that if
	$\delta$ is small enough, then this case does not exist. 
		
	We say a bin is {\em heavy} if it has more than $(1-\epsilon)m/n$ balls.
	Let $Z_t$ be the random variable denoting the number of heavy bins at the
	beginning of the $t$th round. We have
	\begin{equation*}
		\E{Z_t} = \sum_{1 \le k \le n} \E{\ind{X_{t, k} >
		(1-\epsilon)\frac{m}{n}}} > (1-\delta) n.
	\end{equation*}
	$Z_t$ is a non-negative variable in $[0, n]$ and has expected value more
	than $(1-\delta)n$. By Markov's inequality,
	\begin{equation*}
		\Pr[Z_t \le (1-2\delta)n] < \frac{1}{2} \mbox{ and } \Pr[Z_t >
		(1-2\delta)n] > \frac{1}{2}.
	\end{equation*}

	We compute $\E{Z_{t+n/2}}$ from the $Z_t$. If $Z_t > (1-2\delta)n$
	for some constant $\delta < 1/4$, the following hold during $P$, the time
	period between the $t$th round and the $(t+n/2)$th round:

	\begin{enumerate}
		\item At least $(1/2-2\delta)n$ bins in $H_t$ are recycled, 
		\item At least $(1/2-2\delta)(1-\epsilon)m$ balls are recycled,
		\item At least $(1/2-2\delta)n$ bins in $H_t$ are not recycled,
	\end{enumerate}
	where $H_t$ denotes the set of heavy bins at the beginning of the $t$th
	round.

	Given (c), we can find a subset $S_t \subset H_t$ that is composed of
	$(1/2-2\delta)n$ bins in $H_t$ not recycled during $P$.  Note that which
	bins are recycled and which are not depend on the random choices made by the
	system. Hence, $S_t$ varies.  
		
	Next, we derive a lower bound on the expected number of balls in any $S_t$.
	The balls which stay in $S_t$ come from two different sources. There are
	those that stay in $S_t$ at the beginning of the $t$th round, of which
	there are at least $|S_t|(1-\epsilon)m/n$.  There are also those which are
	recycled during $P$, of which there are at least
	$|S_t|(1-\epsilon')|B|/n$ by \Cref{lem:MinBin}, to follow.  Combining the
	two sources, the expected number of balls in $S_t$ is at least
	\begin{equation*}
		\Gamma = (1-\epsilon)\left( \left(\frac{1}{2}-2\delta\right) +
		\left(\frac{1}{2}-2\delta\right)^2(1-\epsilon')\right)m.
	\end{equation*}

	\begin{lemma}\label{lem:MinBin}
	Let $B$ be the multiset of the first $(1/2-2\delta)(1-\epsilon)m$ balls
	recycled during $P$. $B$ is well-defined thanks to 2.\ above. Let $L_i$ be
	the random variable denoting the number of balls in $B$ that land on the
	$i$th bin. For all $\epsilon' > 0$, there exists a $c'$ such
        that if $m \ge c' n \log n$
		\begin{equation*}
			\E{\min\{L_1, L_2, \ldots, L_n\}} \ge (1-\epsilon')|B|/n.
		\end{equation*}
	\end{lemma}
	\begin{proof}
		For $i \in [1,n]$, $\E{L_i} = |B|/n$.
		By Chernoff bounds, 
		$$
			\Pr[|L_i-\E{L_i}| \ge (\epsilon'/2) \E{L_i}] \le \frac{1}{n^2} 
		$$
		for some sufficiently large $c'$. Consequently, by the union bound,
		$$
			\Pr[\min\{L_1, L_2, \ldots, L_n\} \le (1-\epsilon'/2)|B|/n] \le \frac{1}{n}.
		$$
		Because the $L_i$'s are non-negative, we are done. 
	\end{proof}

	Given $\Gamma$, we obtain the following bound:
	\begin{multline*}
		\E{Z_{t+n/2} \given Z_t > (1-2\delta)n} \le |S_t| + \frac{m-\Gamma}{(1-\epsilon)m/n} \\
		\begin{aligned} &= \left(\frac{1}{1-\epsilon} - \left(\frac{1}{2}-\delta\right)^2(1-\epsilon')\right) n \\
		&\approx \left(\frac{3}{4}+\epsilon+\delta\right) n \end{aligned}
	\end{multline*}
	Together with the trivial bound $\E{Z_{t+n/2} \given Z_t \le (1-2\delta)n}
	\le n$,  $\E{Z_{t+n/2}}$ equals
	\begin{multline*}
		\Prob{Z_t \le (1-2\delta)}\E{Z_{t+n/2} \given Z_t \le (1-2\delta)} \\
		+ \Prob{Z_t > (1-2\delta)}\E{Z_{t+n/2} \given Z_t > (1-2\delta)} \\
		\begin{aligned} &< \frac{1}{2} \left( 1 + \left(\frac{3}{4}+\epsilon+\delta\right) \right) n\\
		&\approx \left(\frac{7}{8} + \frac{\epsilon+\delta}{2}\right) n \end{aligned}
	\end{multline*}
		
	This leads to a contradiction if $\epsilon+\delta$ is small enough.  This
	is because we have $\E{Z_t} > (1-\delta)n$ and $\E{Z_{t+n/2}} = \E{Z_t}$,
	because the system is stationary.  As a result, we have a contradiction if
	$\frac{\epsilon+3\delta}{2} < \frac{1}{8}$.

	Combining the results for the two cases, we wish to maximize
	$1+\epsilon^2\delta$ subject to $\epsilon+3\delta < \frac{1}{4}.$ Picking
	$\epsilon = 1/6$ yields the result.
\end{proof}

\Cref{thm:rbuniform} follows from
\Cref{cor:rbuni-upper-constants,lem:rbuni-lower,cor:uniformupperbound}.


%%% Local Variables:
%%% mode: latex
%%% TeX-master: "paper.tex"
%%% End:
