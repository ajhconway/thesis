% !TEX root =  paper.tex

\section{Database Experiments}\label{sec:experiments}
In this section, we consider insertion buffers as they are used in practice. We
demonstrate through simulations as well as experiments on real-world systems,
that the theoretical results in the prior sections hold and can be used to
improve performance.

\subsection{Insertion Buffers in Database Systems}
Many databases cache recently inserted items in RAM so that they can write
items to disk in batches.  Examples include
Azure~\cite{Azure16}, 
DB2~\cite{IBM17},  
Hbase~\cite{Xiang12}, 
Informix~\cite{Informix}, 
InnoDB~\cite{Callaghan11},  
NuDB~\cite{NuDB16},  
Oracle~\cite{Oracle17},  
SAP~\cite{SAP17},  and 
Vertica~\cite{Vertica17}.
They are also used to accelerate inserts in several research prototypes, such
as the buffered Bloom
filter~\cite{CanimLaMi10} and buffered quotient
filter~\cite{BenderFaJo12}.  
By batching updates to disk, these insertion buffers reduce the amortized
number of I/Os per insert, which can substantially improve insertion
throughput.  Facebook claims that the insertion buffer in InnoDB speeds up some
production workloads by a factor of $5$ to $16$, and accelerates some synthetic
benchmarks by up to a factor of $80$~\cite{Callaghan11}.

A motivating factor for the use of insertion buffers is that they can
significantly mitigate the precipitous performance drop that databases can
experience when the data set grows too large to fit in RAM.
\Cref{fig:introbufferinitial} shows the time per 1,000 insertions into a MySQL
database using the InnoDB backend, with and without InnoDB's insertion buffer
enabled.  For the first $200,000$ insertions, the entire database fits in RAM,
and so insertions are fast, even without the insertion buffer.

Once the database grows larger than RAM, insertion performance without the
insertion buffer falls off a cliff.  In fact, once the database reaches 1M
rows, it can perform only about 200 insertions per second, suggesting that the
throughput is limited by the random-I/O performance of the underlying disk.  In
the benchmark with the insertion buffer enabled, on the other hand, performance
degrades by only a small amount.

Based on the performance of the first 1M insertions, it appears that InnoDB's
insertion buffer effectively eliminates the performance cliff that can occur
when the database grows larger than RAM.  This improvement explains the
popularity of insertion buffers in database design.

However, in our experiment, as the database continues to grow, the efficacy of
the insertion buffer declines.  \Cref{fig:introbuffer} shows the time per
10,000 insertions as the database grows to 50M rows.  Although the performance
without the insertion buffer drops more quickly early on, it remains relatively
stable thereafter.  Performance with the insertion buffer, on the other hand,
slowly declines over the course of the benchmark until it is only about a third
faster than without the insertion buffer.  This is well below the $5-80\times$
speedups reported above.

% !TEX root =  paper.tex

\begin{figure*}
{\centering
\ref{intro_legend}
\subfloat[First million insertions]{
\tikzsetnextfilename{intro-buffer-initial}
	\begin{tikzpicture}%[trim axis left, trim axis right]
\begin{axis}[
	width=0.5\textwidth,
	xlabel=Rows inserted (in millions),
	ylabel=Seconds per thousand insertions,
	ylabel near ticks,
	%y label style={at={(axis description cs:0.15,.5)},anchor=south},
	xmin=0,
	xmax=1,
	ymin=0,
	%ymax=100,
	mark repeat={2},
	legend to name=intro_legend,
    legend columns=2,
	]
	%\addplot [mark=*, blue, mark size=2] table [x=insertions, y=lower bound, col sep=comma] {./data/innodb/goldengate_uniform_1.0.csv};
	%\addplot [mark=square*, red, mark size=2] table [x=insertions, y=upper bound, col sep=comma] {./data/innodb/goldengate_uniform_1.0.csv};
	\addplot [mark=triangle*, YellowOrange, mark size=2] table [x expr=\thisrowno{0}/1000000, y expr=\thisrowno{1} / 10000, col sep=comma] {./data/intro/buffer_off/output.txt};
	\addlegendentry{Insertion buffer disabled}
	\addplot [mark=pentagon*, RoyalPurple, mark size=2] table [x expr=\thisrowno{0}/1000000, y expr=\thisrowno{1} / 10000, col sep=comma] {./data/intro/buffer_on/output.txt};
 	\addlegendentry{Insertion buffer enabled}
\end{axis}
\end{tikzpicture}
\label{fig:introbufferinitial}
}
\subfloat[All fifty million insertions] {
\tikzsetnextfilename{intro-buffer}
\begin{tikzpicture}%[trim axis left, trim axis right]
\begin{axis}[
	width=0.5\textwidth,
	xlabel=Rows inserted (in millions),
	ylabel=Seconds per thousand insertions,
	ylabel near ticks,
	%y label style={at={(axis description cs:0.15,.5)},anchor=south},
	xmin=0,
	xmax=50,
	ymin=0,
	%ymax=100,
	mark repeat={5},
	]
	%\addplot [mark=*, blue, mark size=2] table [x=insertions, y=lower bound, col sep=comma] {./data/innodb/goldengate_uniform_1.0.csv};
	%\addplot [mark=square*, red, mark size=2] table [x=insertions, y=upper bound, col sep=comma] {./data/innodb/goldengate_uniform_1.0.csv};
	\addplot [mark=triangle*, YellowOrange, mark size=2, each nth point={10}] table [x expr=\thisrowno{0}/1000000, y expr=\thisrowno{1} / 10000, col sep=comma] {./data/intro/buffer_off/output.txt};
	\addplot [mark=pentagon*, RoyalPurple, mark size=2, each nth point={10}] table [x expr=\thisrowno{0}/1000000, y expr=\thisrowno{1} / 10000, col sep=comma] {./data/intro/buffer_on/output.txt};
\end{axis}
\end{tikzpicture}
\label{fig:introbuffer}
}
	\caption{The cost of inserting batches of rows into an empty
          table in \innodb{} with and without the insertion
          buffer. The rows are inserted in batches of 10,000 to avoid
          slowdown in parsing, and the keys are distributed
          uniformly. After 1M insertions, the buffered version takes
          $12.3\%$ as long as the unbuffered version (measured over
          50000 insertions); after 50M insertions, the advantage is
          reduced so that the buffered version takes $68.3\%$ of
          the time of the unbuffered. (Lower is better)}
}
\end{figure*}



%%% Local Variables:
%%% mode: latex
%%% TeX-master: "paper.tex"
%%% End:


As these experiments show, it can be difficult to extrapolate from small
examples the performance gains that insertion buffers can provide for large
databases. Therefore, it is no wonder that reported speedups from insertion
buffers vary wildly from as little as $2\times$ to as high as
$80\times$~\cite{Callaghan11}.  Some have even suggested that insertion buffers
may provide many of the benefits of write-optimization~\cite{Callaghan10},
i.e., that insertion buffers can bring the performance of \btrees{} up to that
of LSM-trees~\cite{ONeilChGa96}, COLAs~\cite{BenderFaFi07}, Fractal
Trees~\cite{Tokutek14}, xDicts~\cite{BrodalDeFi10}, or
\betrees{}~\cite{BrodalFa03}.

\subsection{Experimental Validation}
Here we validate our theoretical study of insertion buffers by showing that our
analysis above can have a material impact on the performance of databases with
insertions buffers.  We simulated workloads of random insertions to a \btree{},
with varying distributions on the inserted keys.  We found that, as predicted,
the performance was independent of the input distribution and closely matched
the performance predicted by our theorems.

We then ran workloads of random insertions into InnoDB and measured the average
batch size of flushes from its insertion buffer.  InnoDB implements a variant
of the random-item flushing strategy.  We modified it to implement the
golden-gate flushing strategy.  Despite the additional complexities of InnoDB's
insertion buffer implementation, we found that performance closely tracked our
theoretical predictions and was independent of the distribution of inserted
keys.  We also found that the golden-gate flushing strategy improved InnoDB's
flushing rate by about 30\% over the course of our benchmark.

Our analysis explains why insertion buffers can provide dramatic speedups for
small databases, but only small gains are available as the database grows.  Our
results also provide useful guidance to implementers about which flushing
strategy will provide the most performance improvement.

Our results also show that insertion buffers cannot deliver the same asymptotic
performance improvements that are possible with write-optimized data
structures, such as LSM-trees and B$^\epsilon$-trees.

\subsecput{exp-deployed}{Insertion-Buffer Background}

This section describes insertion buffers are actually implemented and used in
deployed systems and recent research prototypes.

\paragraph{SAP:} The SAP IQ database supports an in-memory row-level versioning
(RLV) store, and insertions are performed to the RLV store and later merged
into the main on-disk store~\cite{SAP17}.  

\paragraph{NuDB:}  The NuDB SSD-based key-value store buffers all insertions in
memory, and later flushes it to SSD~\cite{NuDB16}.  Flushes occur at least once
per second, or more often if insertion activity causes the in-memory buffer to
fill.

\paragraph{Buffered Bloom and quotient filters:}  Bloom filters are known to
have poor locality for both inserts and lookups.  The buffered Bloom
filter~\cite{CanimLaMi10} improves the performance of insertions to a Bloom
filter on SSD by buffering the updates in RAM.  The on-disk Bloom filter is
divided into pages, and each page has a buffer of updates in RAM.  When a
page's buffer fills, the buffered changes are written to the page.  

The buffered quotient filter stores newly inserted items in an in-memory
quotient filter~\cite{BenderFaJo12,benderadaptivebloom}.  When the in-memory
quotient filter fills, its entire contents are flushed to the on-disk quotient
filter.

\paragraph{InnoDB:}  The InnoDB~\cite{Oracle17a} \btree{} implementation used
in the MySQL~\cite{Oracle17b} and MariaDB~\cite{Foundation17} relational
database systems includes an insertion buffer.

Our experiments in this paper focus on InnoDB as an archetypal and open-source
implementation of an insertion buffer, so we describe it in detail.

InnoDB structures its insertion buffer as a \btree{}.  When the insertion
buffer becomes full, it selects the items to be flushed by performing a random
walk from the root to a leaf.  The random walk is performed by selecting, at
each step, uniformly randomly from among the children of the current node.
Once it gets to a leaf, it it picks a single item to insert into the on-disk
\btree{}.  This item, along with any other items in the insertion buffer that
belong in that leaf, are inserted into the leaf and removed from the insertion
buffer.

InnoDB's insertion buffer is complicated in several ways.  First, the size of
the insertion buffer changes over time, as InnoDB allocates more or less space
to other buffers and caches.

InnoDB also has a leaf cache.  Whenever a leaf is brought into cache for any
reason, all inserts to that leaf that are currently in the insertion buffer are
immediately applied to the leaf, and any future inserts to that leaf also skip
the insertion buffer as long as the leaf remains in cache.

Finally, it performs some flushing when the buffer is not full. Roughly every
second, InnoDB performs a small amount of background flushing. Moreover, it
prematurely flushes its buffer to a leaf when it calculates that such a flush
will cause the leaf to split. We hypothesize that this feature exists to
simplify the transactional system.

\subsecput{exp-insertion}{Leaf Probabilities in \btrees{}}
In \Cref{sec:uniform}, we established that, on insertion, the leaf
probabilities are nearly uniform.  We empirically verify this uniformity
property by simulating insertions into the leaves of a \btree{}. We insert
real-valued keys i.i.d.\ according to uniform, Pareto (real-valued Zipfian) and
normal distributions; the leaves of the \btree{} split when they are full, and
we measure the ratio of the maximal weight leaf to $1/n$.
\Cref{lem:uniform-leaves} tells us that this ratio should be asymptotically at
most constant, but as \Cref{fig:simalpha} shows, our experimental analysis
shows further that this constant is generally less than 2. Because leaves
generally split in 2, this makes some intuitive sense.

% !TEX root =  paper.tex

\begin{figure}[h!]
{\centering
\hspace*{33pt}\ref{full_distro_legend}\\
\tikzsetnextfilename{insertion-sim-max}
	\begin{tikzpicture}%[trim axis left, trim axis right]
		\begin{axis}[
			%title=Uniform Key Distribution,
			xlabel=Rows inserted in millions,
			xlabel near ticks,
			ylabel=Ratio of maximum leaf weight to $1/n$,
			ylabel near ticks,
			xmin=0,
			xmax=5,
			ymin=1,
			mark repeat={25},
			legend to name=full_distro_legend,
			legend columns=2,
			]
			\addplot [mark=*, Blue, mark size=2] table [x=insertions, y=max_alpha, col sep=comma] {./data/simulations/noop_uni.csv};
			\addlegendentry{Uniform}
			\addplot [mark=triangle*, Cyan, mark size=2] table [x=insertions, y=max_alpha, col sep=comma] {./data/simulations/noop_par5.csv};
			\addlegendentry{Pareto $\alpha=0.5$}
			\addplot [mark=square*, Plum, mark size=2] table [x=insertions, y=max_alpha, col sep=comma] {./data/simulations/noop_par1.csv};
			\addlegendentry{Pareto $\alpha=1.0$}
			\addplot [mark=diamond*, Magenta, mark size=2] table [x=insertions, y=max_alpha, col sep=comma] {./data/simulations/noop_par2.csv};
			\addlegendentry{Pareto $\alpha=2.0$}
			\addplot [mark=x, YellowOrange, mark size=2] table [x=insertions, y=max_alpha, col sep=comma] {./data/simulations/noop_norm.csv};
			\addlegendentry{Normal}
			%\legend{Uniform, Zipf $\alpha = 0.5$, Zipf $\alpha = 1.0$, Zipf $\alpha = 1.5$, Zipf $\alpha = 2.0$}
		\end{axis}
	\end{tikzpicture}
\caption{Deviation of the maximum weight leaf from uniform in simulation}
\label{fig:simalpha}
}
\end{figure}

\begin{figure}
{\centering
\hspace*{35pt}\ref{full_distro_legend}\\
\subfloat[Maximum leaf weight relative to uniform]{
\tikzsetnextfilename{insertion-inno-max}
\begin{tikzpicture}%[trim axis left, trim axis right]
\begin{axis}[
	%title=Uniform Key Distribution,
    width=0.47\textwidth,
	xlabel=Rows inserted in millions,
	xlabel near ticks,
	ylabel near ticks,
	ylabel=Ratio of maximum leaf weight to $1/n$,
	xmin=0,
	xmax=5,
	ymin=1,
	mark repeat={25},
	legend columns=2,
	legend to name=distro_legend
	]
	\addplot [mark=*, Blue, mark size=2] table [x=insertions, y=alpha, col sep=comma] {./data/innodb/randomball_uniform_1.0.csv};
	\addlegendentry{Uniform}
	\addplot [mark=triangle*, Cyan, mark size=2] table [x=insertions, y=alpha, col sep=comma] {./data/innodb/randomball_pareto_0.5.csv};
	\addlegendentry{Pareto $\alpha=0.5$}
	\addplot [mark=square*, Plum, mark size=2] table [x=insertions, y=alpha, col sep=comma] {./data/innodb/randomball_pareto_1.0.csv};
	\addlegendentry{Pareto $\alpha=1.0$}
	\addplot [mark=diamond*, Magenta, mark size=2] table [x=insertions, y=alpha, col sep=comma] {./data/innodb/randomball_normal_1000.0.csv};
	\addlegendentry{Pareto $\alpha=2.0$}
	\addplot [mark=x, YellowOrange, mark size=2] table [x=insertions, y=alpha, col sep=comma] {./data/innodb/randomball_normal_1000.0.csv};
	\addlegendentry{Normal}
	%\legend{Uniform, Zipf $\alpha = 0.5$, Zipf $\alpha = 1.0$, Zipf $\alpha = 1.5$, Zipf $\alpha = 2.0$}
\end{axis}
\end{tikzpicture}
\label{fig:alpha}
}\hfill
\subfloat[95th percentile leaf weight relative to uniform]{
\tikzsetnextfilename{insertion-inno-95}
\begin{tikzpicture}%[trim axis left, trim axis right]
\begin{axis}[
	%title=Uniform Key Distribution,
	width=0.47\textwidth,
	xlabel=Rows inserted in millions,
	xlabel near ticks,
	ylabel near ticks,
	ylabel=Ratio of 95th percentile leaf weight to $1/n$,
	xmin=0,
	xmax=5,
	ymin=1,
	mark repeat={25}
	]
	\addplot [mark=*, Blue, mark size=2] table [x=insertions, y=perc, col sep=comma] {./data/innodb/randomball_uniform_1.0.csv};
	\addplot [mark=triangle*, Cyan, mark size=2] table [x=insertions, y=perc, col sep=comma] {./data/innodb/randomball_pareto_0.5.csv};
	\addplot [mark=square*, Plum, mark size=2] table [x=insertions, y=perc, col sep=comma] {./data/innodb/randomball_pareto_1.0.csv};
	\addplot [mark=diamond*, Magenta, mark size=2] table [x=insertions, y=perc, col sep=comma] {./data/innodb/randomball_pareto_2.0.csv};
	\addplot [mark=x, YellowOrange, mark size=2] table [x=insertions, y=perc, col sep=comma] {./data/innodb/randomball_normal_1000.0.csv};
	%\legend{Uniform, Zipf $\alpha = 0.5$, Zipf $\alpha = 1.0$, Zipf $\alpha = 1.5$, Zipf $\alpha = 2.0$}
\end{axis}
\end{tikzpicture}
\label{fig:perc}
}
	\caption{Deviation of the maximum and 95th percentile weight leaves from uniform as observed in \innodb{}.}\label{fig:deviation}
}
\end{figure}



%%% Local Variables:
%%% mode: latex
%%% TeX-master: "paper.tex"
%%% End:


We also verify the these results using the \innodb{} storage engine. We insert
5 million rows into a database using uniform, Pareto and normal distributions
on the keys. the results are summarized in \cref{fig:alpha}. The maximum ratio
does not exceed 2.3, and the 95th percentile ratio does exceed 1.6. Thus the
distribution of the keys to the leaves is in fact almost uniform.

\subsecput{exp-synthetic}{Simulating Insertion Buffers}

The ball-and-bins models described above are based on a static leaf structure.
However, in practice inserting into a database causes the leaf structure (the
number and probability distribution of bins in the model) to change. However,
we can still perform the same strategies, and by simulating an insertion buffer
in front of a database, we can compare their efficiency as well as verify that
much of the static analysis empirically applies to the dynamic system.

We insert real-valued keys into the simulation according to one of several
distributions of varying skewness: uniform on $[0,1000]$, Pareto with parameter
$\alpha=\{0.5,1.0,2.0\}$, and uniform centered at 0, with standard deviation
$1000$. We have a buffer which stores 2,500 keys; when it fills we choose a
leaf according to the chosen strategy and flush all the buffered keys destined
to it. Initially we have one leaf, and the leaves split when they exceed 160
keys, as uniformly as possible. 

As shown in \Cref{fig:synth}, the key distribution doesn't affect the recycle
rate of the insertion buffer, and as the number of leaves gets larger, the
recycle rate decreases. Generally fullest bin does better than golden gate, and
golden gate does better than random ball. Demonstrated with the normal
distribution (all distributions perform very similarly), \Cref{fig:synth-ratio}
shows that golden gate initially outperforms random ball by about 30\%, which
then decreases as the number of bins grows.

% !TEX root =  paper.tex

\begin{figure*}[p]
{\centering
	\ref{synth-legend}\\
	\subfloat[Uniform key distribution]{
		\tikzsetnextfilename{synthetic-uniform}
		\begin{tikzpicture}%[trim axis left, trim axis right]
		\begin{axis}[
				width=0.47\textwidth,
				height=0.31\textheight,
				xlabel=Rows inserted in millions,
				xlabel near ticks,
				ylabel near ticks,
				ylabel style={align=center, text width=0.27\textheight},
				ylabel=Recycle rate (last {50,000} insertions),
				xmin=0.05,
				xmax=5,
				ymin=0,
				ymax=50,
				mark repeat={50},
				restrict y to domain=0:50,
				legend to name=synth-legend,
				legend columns = 3,
				]
				\addplot [mark=*, blue, mark size=2] table [x=insertions, y expr=50000.0/\thisrow{flushes}, col sep=comma] {./data/simulations/gg_uni.csv};
				\addlegendentry{Golden Gate}
				\addplot [mark=square*, red, mark size=2] table [x=insertions, y expr=50000.0/\thisrow{flushes}, col sep=comma] {./data/simulations/rb_uni.csv};
				\addlegendentry{Random Ball}
				\addplot [mark=triangle*, green, mark size=2] table [x=insertions, y expr=50000.0/\thisrow{flushes}, col sep=comma] {./data/simulations/fb_uni.csv};
				\addlegendentry{Fullest Bin}
			\end{axis}
		\end{tikzpicture}
		\label{fig:synth-uni}
	}\hfill
	\subfloat[Pareto-0.5 key distribution]{
		\tikzsetnextfilename{synthetic-pareto-5}
		\begin{tikzpicture}%[trim axis left, trim axis right]
			\begin{axis}[
				width=0.47\textwidth,
				height=0.31\textheight,
				xlabel=Rows inserted in millions,
				xlabel near ticks,
				ylabel near ticks,
				ylabel style={align=center, text width=0.27\textheight},
				ylabel=Recycle rate (last {50,000} insertions),
				xmin=0.05,
				xmax=5,
				ymin=0,
				ymax=50,
				mark repeat={50},
				restrict y to domain=0:50,
				]
				\addplot [mark=*, blue, mark size=2] table [x=insertions, y expr=50000.0/\thisrow{flushes}, col sep=comma] {./data/simulations/gg_par5.csv};
				\addplot [mark=square*, red, mark size=2] table [x=insertions, y expr=50000.0/\thisrow{flushes}, col sep=comma] {./data/simulations/rb_par5.csv};
				\addplot [mark=triangle*, green, mark size=2] table [x=insertions, y expr=50000.0/\thisrow{flushes}, col sep=comma] {./data/simulations/fb_par5.csv};
			\end{axis}
		\end{tikzpicture}
		\label{fig:synth-par5}
	}\\
	\subfloat[Pareto-1 key distribution]{
		\tikzsetnextfilename{synthetic-pareto-1}
		\begin{tikzpicture}%[trim axis left, trim axis right]
			\begin{axis}[
				width=0.47\textwidth,
				height=0.31\textheight,
				xlabel=Rows inserted in millions,
				xlabel near ticks,
				ylabel near ticks,
				ylabel style={align=center, text width=0.27\textheight},
				ylabel=Recycle rate (last {50,000} insertions),
				xmin=0.05,
				xmax=5,
				ymin=0,
				ymax=50,
				mark repeat={50},
				restrict y to domain=0:50,
				]
				\addplot [mark=*, blue, mark size=2] table [x=insertions, y expr=50000.0/\thisrow{flushes}, col sep=comma] {./data/simulations/gg_par1.csv};
				\addplot [mark=square*, red, mark size=2] table [x=insertions, y expr=50000.0/\thisrow{flushes}, col sep=comma] {./data/simulations/rb_par1.csv};
				\addplot [mark=triangle*, green, mark size=2] table [x=insertions, y expr=50000.0/\thisrow{flushes}, col sep=comma] {./data/simulations/fb_par1.csv};
			\end{axis}
		\end{tikzpicture}
	}\hfill
	\subfloat[Pareto-2 key distribution]{
		\tikzsetnextfilename{synthetic-pareto-2}
		\begin{tikzpicture}%[trim axis left, trim axis right]
			\begin{axis}[
				width=0.47\textwidth,
				height=0.31\textheight,
				xlabel=Rows inserted in millions,
				xlabel near ticks,
				ylabel near ticks,
				ylabel style={align=center, text width=0.27\textheight},
				ylabel=Recycle rate (last {50,000} insertions),
				xmin=0.05,
				xmax=5,
				ymin=0,
				ymax=50,
				mark repeat={50},
				restrict y to domain=0:50,
				]
				\addplot [mark=*, blue, mark size=2] table [x=insertions, y expr=50000.0/\thisrow{flushes}, col sep=comma] {./data/simulations/gg_par2.csv};
				\addplot [mark=square*, red, mark size=2] table [x=insertions, y expr=50000.0/\thisrow{flushes}, col sep=comma] {./data/simulations/rb_par2.csv};
				\addplot [mark=triangle*, green, mark size=2] table [x=insertions, y expr=50000.0/\thisrow{flushes}, col sep=comma] {./data/simulations/fb_par2.csv};
			\end{axis}
		\end{tikzpicture}
		\label{fig:synth-par2}
	}\\
	\vspace*{-0.015\textheight}
	\subfloat[Normal key distribution]{
		\tikzsetnextfilename{synthetic-normal}
		\begin{tikzpicture}%[trim axis left, trim axis right]
			\begin{axis}[
				width=0.47\textwidth,
				height=0.31\textheight,
				xlabel=Rows inserted in millions,
				xlabel near ticks,
				ylabel near ticks,
				ylabel style={align=center, text width=0.27\textheight},
				ylabel=Recycle rate (last {50,000} insertions),
				xmin=0.05,
				xmax=5,
				ymin=0,
				ymax=50,
				mark repeat={50},
				restrict y to domain=0:50,
				]
				\addplot [mark=*, blue, mark size=2] table [x=insertions, y expr=50000.0/\thisrow{flushes}, col sep=comma] {./data/simulations/gg_norm.csv};
				\addplot [mark=square*, red, mark size=2] table [x=insertions, y expr=50000.0/\thisrow{flushes}, col sep=comma] {./data/simulations/rb_norm.csv};
				\addplot [mark=triangle*, green, mark size=2] table [x=insertions, y expr=50000.0/\thisrow{flushes}, col sep=comma] {./data/simulations/fb_norm.csv};
			\end{axis}
		\end{tikzpicture}
		\label{fig:synth-norm}
	}\hfill
	\subfloat[Ratio of golden gate to random ball (normally distributed keys)]{
		\tikzsetnextfilename{synthetic-norm-ratio}
		\pgfplotstableread[col sep = comma]{./data/simulations/gg_norm.csv}\goldengate
		\pgfplotstableread[col sep = comma]{./data/simulations/rb_norm.csv}\randomball
		\pgfplotstablecreatecol[copy column from table={\randomball}{flushes}] {rb_flushes} {\goldengate}
		\begin{tikzpicture}%[trim axis left, trim axis right]
			\begin{axis}[
				width=0.47\textwidth,
				height=0.31\textheight,
				xlabel=Rows inserted in millions,
				xlabel near ticks,
				ylabel near ticks,
				ylabel style={align=center, text width=0.30\textheight},
				ylabel=Recycle rate ratio (last {5,000} insertions),
				xmin=0.05,
				xmax=5,
				ymin=0.8,
				ymax=2,
				mark repeat={50},
				]
				\addplot [mark=*, Plum, no marks] table [x=insertions, y expr=\thisrow{rb_flushes}/\thisrow{flushes}] \goldengate;
			\end{axis}
		\end{tikzpicture}
		\label{fig:synth-ratio}
	}
	\caption{\label{fig:synth}Simulated results with various key distributions and recycling strategies. (Higher is better)}
}
\end{figure*}



%%% Local Variables:
%%% mode: latex
%%% TeX-master: "paper.tex"
%%% End:


\subsecput{exp-innodb}{Real-World Performance (InnoDB)}

In this section, we empirically the performance of insertion buffers in
\innodb{}, the default storage engine in \mysql{}.

Analogously to the experiments in \secref{exp-synthetic}, we insert rows into
the \mysql{} database, and after every 10000 insertions, we check the ``merge
ratio'' reported by \innodb{}.  This is the number of rows merged into the
database from the buffer during each buffer flush, and corresponds to the
recycling rate in the balls and bins model. We also check the reported memory
allocated to the buffer, which allows us to control for memory usage. 

% !TEX root =  paper.tex

\begin{figure*}
{\centering
~\ref{innodb-legend}~\\
\subfloat[Uniform key distribution]{
\tikzsetnextfilename{innodb-uniform}
\begin{tikzpicture}%[trim axis left, trim axis right]
\begin{axis}[
	width=0.48\textwidth,
	height=0.31\textheight,
	xlabel=Rows inserted in millions,
	xlabel near ticks,
	ylabel near ticks,
	ylabel style={align=center, text width=0.27\textheight},
	ylabel=Recycle rate (last {10,000} insertions),
	xmin=0,
	xmax=5,
	ymin=0,
	ymax=50,
	mark repeat={50},
	legend to name=innodb-legend,
	legend columns=3,
	]
	%\addplot [mark=*, blue, mark size=2] table [x=insertions, y=lower bound, col sep=comma] {./data/innodb/goldengate_uniform_1.0.csv};
	%\addplot [mark=square*, red, mark size=2] table [x=insertions, y=upper bound, col sep=comma] {./data/innodb/goldengate_uniform_1.0.csv};
	\addplot [mark=*, blue, mark size=2] table [x=insertions, y=observed, col sep=comma] {./data/innodb/goldengate_uniform_1.0.csv};
	\addlegendentry{Golden Gate}
	\addplot [mark=square*, red, mark size=2] table [x=insertions, y=observed, col sep=comma] {./data/innodb/randomball_uniform_1.0.csv};
	\addlegendentry{Random Ball (default)}
\end{axis}
\end{tikzpicture}
\label{fig:innodbuni}
}\hfill
\subfloat[Pareto $\alpha=0.5$ key distribution]{
\tikzsetnextfilename{innodb-pareto5}
\begin{tikzpicture}%[trim axis left, trim axis right]
\begin{axis}[
	%title=Uniform Key Distribution,
	width=0.48\textwidth,
	height=0.31\textheight,
	xlabel=Rows inserted in millions,
	xlabel near ticks,
	ylabel near ticks,
	ylabel style={align=center, text width=0.27\textheight},
	ylabel=Recycle rate (last {10,000} insertions),
	xmin=0,
	xmax=5,
	ymin=0,
	ymax=50,
	mark repeat={50},
	]
	%\addplot [mark=*, blue, mark size=2] table [x=insertions, y=lower bound, col sep=comma] {./data/innodb/goldengate_pareto_0.5.csv};
	%\addplot [mark=square*, red, mark size=2] table [x=insertions, y=upper bound, col sep=comma] {./data/innodb/goldengate_pareto_0.5.csv};
	\addplot [mark=*, blue, mark size=2] table [x=insertions, y=observed, col sep=comma] {./data/innodb/goldengate_pareto_0.5.csv};
	\addplot [mark=square*, red, mark size=2] table [x=insertions, y=observed, col sep=comma] {./data/innodb/randomball_pareto_0.5.csv};
\end{axis}
\end{tikzpicture}
\label{fig:innodbpar0.5}
}\\
\subfloat[Pareto $\alpha=1$ key distribution]{
\tikzsetnextfilename{innodb-pareto1}
\begin{tikzpicture}%[trim axis left, trim axis right]
\begin{axis}[
	%title=Uniform Key Distribution,
	width=0.48\textwidth,
	height=0.31\textheight,
	xlabel=Rows inserted in millions,
	xlabel near ticks,
	ylabel near ticks,
	ylabel style={align=center, text width=0.27\textheight},
	ylabel=Recycle rate (last {10,000} insertions),
	xmin=0,
	xmax=5,
	ymin=0,
	ymax=50,
	mark repeat={50},
	]
	%\addplot [mark=*, blue, mark size=2] table [x=insertions, y=lower bound, col sep=comma] {./data/innodb/goldengate_pareto_1.0.csv};
	%\addplot [mark=square*, red, mark size=2] table [x=insertions, y=upper bound, col sep=comma] {./data/innodb/goldengate_pareto_1.0.csv};
	\addplot [mark=*, blue, mark size=2] table [x=insertions, y=observed, col sep=comma] {./data/innodb/goldengate_pareto_1.0.csv};
	\addplot [mark=square*, red, mark size=2] table [x=insertions, y=observed, col sep=comma] {./data/innodb/randomball_pareto_1.0.csv};
\end{axis}
\end{tikzpicture}
\label{fig:innodbpar1}
}\hfill
\subfloat[Pareto $\alpha=2$ key distribution]{
\tikzsetnextfilename{innodb-pareto2}
\begin{tikzpicture}%[trim axis left, trim axis right]
\begin{axis}[
	%title=Uniform Key Distribution,
	width=0.48\textwidth,
	height=0.31\textheight,
	xlabel=Rows inserted in millions,
	xlabel near ticks,
	ylabel near ticks,
	ylabel style={align=center, text width=0.27\textheight},
	ylabel=Recycle rate (last {10,000} insertions),
	xmin=0,
	xmax=5,
	ymin=0,
	ymax=50,
	mark repeat={50},
	]
	%\addplot [mark=*, blue, mark size=2] table [x=insertions, y=lower bound, col sep=comma] {./data/innodb/goldengate_pareto_2.0.csv};
	%\addplot [mark=square*, red, mark size=2] table [x=insertions, y=upper bound, col sep=comma] {./data/innodb/goldengate_pareto_2.0.csv};
	\addplot [mark=*, blue, mark size=2] table [x=insertions, y=observed, col sep=comma] {./data/innodb/goldengate_pareto_2.0.csv};
	\addplot [mark=square*, red, mark size=2] table [x=insertions, y=observed, col sep=comma] {./data/innodb/randomball_pareto_2.0.csv};
\end{axis}
\end{tikzpicture}
\label{fig:innodbpar2}
}\\
\subfloat[Normal key distribution]{
\tikzsetnextfilename{innodb-normal}
\begin{tikzpicture}%[trim axis left, trim axis right]
\begin{axis}[
	width=0.48\textwidth,
	height=0.31\textheight,
	xlabel=Rows inserted in millions,
	xlabel near ticks,
	ylabel near ticks,
	ylabel style={align=center, text width=0.27\textheight},
	ylabel=Recycle rate (last {10,000} insertions),
	xmin=0,
	xmax=5,
	ymin=0,
	ymax=50,
	mark repeat={50},
	]
	%\addplot [mark=*, blue, mark size=2] table [x=insertions, y=lower bound, col sep=comma] {./data/innodb/goldengate_uniform_1.0.csv};
	%\addplot [mark=square*, red, mark size=2] table [x=insertions, y=upper bound, col sep=comma] {./data/innodb/goldengate_uniform_1.0.csv};
	\addplot [mark=*, blue, mark size=2] table [x=insertions, y=observed, col sep=comma] {./data/innodb/goldengate_normal_1000.0.csv};
	\addplot [mark=square*, red, mark size=2] table [x=insertions, y=observed, col sep=comma] {./data/innodb/randomball_normal_1000.0.csv};
\end{axis}
\end{tikzpicture}
\label{fig:innodbnorm}
}\hfill
\subfloat[Buffer size and recycle rate]{
\tikzsetnextfilename{innodb-buffer-size}
\begin{tikzpicture}%[trim axis left, trim axis right]
\begin{axis}[
	%title=Uniform Key Distribution,
	width=0.48\textwidth,
	height=0.31\textheight,
	xlabel=Buffer size in MiB,
	xlabel near ticks,
	ylabel near ticks,
	ylabel style={align=center, text width=0.27\textheight},
	ylabel=Recycle rate (last million insertions),
	xmin=0,
	xmax=128,
	ymin=0,
	]
	\addplot [mark=*, Plum, mark size=2] table [x=Buffer, y=Rate, col sep=comma] {./data/innodb_buffer_size.csv};
\end{axis}
\end{tikzpicture}
\label{fig:innodbbuffer}
}

\caption{\innodb{} Insertion buffer recycle rates for various key distributions and memory sizes. (Higher is better)}}
\end{figure*}



%%% Local Variables:
%%% mode: latex
%%% TeX-master: "paper.tex"
%%% End:


The keys of the rows are i.i.d.\ according to the same real-valued probability
distributions as in \secref{exp-synthetic}: uniform on $[0,1000000]$, Pareto
with parameter $\alpha = \{0.5,1,2\}$, and normal centered at 0 with standard
deviation $1000$. The results for the different distributions are shown in
\Cref{fig:innodbuni,fig:innodbpar0.5,fig:innodbpar1,fig:innodbpar2,fig:innodbnorm}.
The structure of the plot generally does not depend on the key distribution,
and while there is more noise, the overall picture is similar to the plots in
\Cref{fig:synth}. 

If we were to hold the number of leaves roughly constant and change the buffer
size, \Cref{lem:unilowerbound} suggests that the relationship with recycle rate
would be roughly linear. To test this, we ran the above experiment with buffer
sizes from 8mb to 128mb in 2mb increments. We performed 11 million insertions
with uniformly distributed keys each time, and then took the average recycle
rate for the last million rows. As demonstrated in \Cref{fig:innodbbuffer}, the
resulting plot is approximately linear.

%%% Local Variables: 
%%% mode: latex
%%% TeX-master: "paper"
%%% End: 
