% !TEX root =  paper.tex

\begin{figure*}
{\centering
~\ref{innodb-legend}~\\
\subfloat[Uniform key distribution]{
\tikzsetnextfilename{innodb-uniform}
\begin{tikzpicture}%[trim axis left, trim axis right]
\begin{axis}[
	width=0.48\textwidth,
	height=0.31\textheight,
	xlabel=Rows inserted in millions,
	xlabel near ticks,
	ylabel near ticks,
	ylabel style={align=center, text width=0.27\textheight},
	ylabel=Recycle rate (last {10,000} insertions),
	xmin=0,
	xmax=5,
	ymin=0,
	ymax=50,
	mark repeat={50},
	legend to name=innodb-legend,
	legend columns=3,
	]
	%\addplot [mark=*, blue, mark size=2] table [x=insertions, y=lower bound, col sep=comma] {./data/innodb/goldengate_uniform_1.0.csv};
	%\addplot [mark=square*, red, mark size=2] table [x=insertions, y=upper bound, col sep=comma] {./data/innodb/goldengate_uniform_1.0.csv};
	\addplot [mark=*, blue, mark size=2] table [x=insertions, y=observed, col sep=comma] {./data/innodb/goldengate_uniform_1.0.csv};
	\addlegendentry{Golden Gate}
	\addplot [mark=square*, red, mark size=2] table [x=insertions, y=observed, col sep=comma] {./data/innodb/randomball_uniform_1.0.csv};
	\addlegendentry{Random Ball (default)}
\end{axis}
\end{tikzpicture}
\label{fig:innodbuni}
}\hfill
\subfloat[Pareto $\alpha=0.5$ key distribution]{
\tikzsetnextfilename{innodb-pareto5}
\begin{tikzpicture}%[trim axis left, trim axis right]
\begin{axis}[
	%title=Uniform Key Distribution,
	width=0.48\textwidth,
	height=0.31\textheight,
	xlabel=Rows inserted in millions,
	xlabel near ticks,
	ylabel near ticks,
	ylabel style={align=center, text width=0.27\textheight},
	ylabel=Recycle rate (last {10,000} insertions),
	xmin=0,
	xmax=5,
	ymin=0,
	ymax=50,
	mark repeat={50},
	]
	%\addplot [mark=*, blue, mark size=2] table [x=insertions, y=lower bound, col sep=comma] {./data/innodb/goldengate_pareto_0.5.csv};
	%\addplot [mark=square*, red, mark size=2] table [x=insertions, y=upper bound, col sep=comma] {./data/innodb/goldengate_pareto_0.5.csv};
	\addplot [mark=*, blue, mark size=2] table [x=insertions, y=observed, col sep=comma] {./data/innodb/goldengate_pareto_0.5.csv};
	\addplot [mark=square*, red, mark size=2] table [x=insertions, y=observed, col sep=comma] {./data/innodb/randomball_pareto_0.5.csv};
\end{axis}
\end{tikzpicture}
\label{fig:innodbpar0.5}
}\\
\subfloat[Pareto $\alpha=1$ key distribution]{
\tikzsetnextfilename{innodb-pareto1}
\begin{tikzpicture}%[trim axis left, trim axis right]
\begin{axis}[
	%title=Uniform Key Distribution,
	width=0.48\textwidth,
	height=0.31\textheight,
	xlabel=Rows inserted in millions,
	xlabel near ticks,
	ylabel near ticks,
	ylabel style={align=center, text width=0.27\textheight},
	ylabel=Recycle rate (last {10,000} insertions),
	xmin=0,
	xmax=5,
	ymin=0,
	ymax=50,
	mark repeat={50},
	]
	%\addplot [mark=*, blue, mark size=2] table [x=insertions, y=lower bound, col sep=comma] {./data/innodb/goldengate_pareto_1.0.csv};
	%\addplot [mark=square*, red, mark size=2] table [x=insertions, y=upper bound, col sep=comma] {./data/innodb/goldengate_pareto_1.0.csv};
	\addplot [mark=*, blue, mark size=2] table [x=insertions, y=observed, col sep=comma] {./data/innodb/goldengate_pareto_1.0.csv};
	\addplot [mark=square*, red, mark size=2] table [x=insertions, y=observed, col sep=comma] {./data/innodb/randomball_pareto_1.0.csv};
\end{axis}
\end{tikzpicture}
\label{fig:innodbpar1}
}\hfill
\subfloat[Pareto $\alpha=2$ key distribution]{
\tikzsetnextfilename{innodb-pareto2}
\begin{tikzpicture}%[trim axis left, trim axis right]
\begin{axis}[
	%title=Uniform Key Distribution,
	width=0.48\textwidth,
	height=0.31\textheight,
	xlabel=Rows inserted in millions,
	xlabel near ticks,
	ylabel near ticks,
	ylabel style={align=center, text width=0.27\textheight},
	ylabel=Recycle rate (last {10,000} insertions),
	xmin=0,
	xmax=5,
	ymin=0,
	ymax=50,
	mark repeat={50},
	]
	%\addplot [mark=*, blue, mark size=2] table [x=insertions, y=lower bound, col sep=comma] {./data/innodb/goldengate_pareto_2.0.csv};
	%\addplot [mark=square*, red, mark size=2] table [x=insertions, y=upper bound, col sep=comma] {./data/innodb/goldengate_pareto_2.0.csv};
	\addplot [mark=*, blue, mark size=2] table [x=insertions, y=observed, col sep=comma] {./data/innodb/goldengate_pareto_2.0.csv};
	\addplot [mark=square*, red, mark size=2] table [x=insertions, y=observed, col sep=comma] {./data/innodb/randomball_pareto_2.0.csv};
\end{axis}
\end{tikzpicture}
\label{fig:innodbpar2}
}\\
\subfloat[Normal key distribution]{
\tikzsetnextfilename{innodb-normal}
\begin{tikzpicture}%[trim axis left, trim axis right]
\begin{axis}[
	width=0.48\textwidth,
	height=0.31\textheight,
	xlabel=Rows inserted in millions,
	xlabel near ticks,
	ylabel near ticks,
	ylabel style={align=center, text width=0.27\textheight},
	ylabel=Recycle rate (last {10,000} insertions),
	xmin=0,
	xmax=5,
	ymin=0,
	ymax=50,
	mark repeat={50},
	]
	%\addplot [mark=*, blue, mark size=2] table [x=insertions, y=lower bound, col sep=comma] {./data/innodb/goldengate_uniform_1.0.csv};
	%\addplot [mark=square*, red, mark size=2] table [x=insertions, y=upper bound, col sep=comma] {./data/innodb/goldengate_uniform_1.0.csv};
	\addplot [mark=*, blue, mark size=2] table [x=insertions, y=observed, col sep=comma] {./data/innodb/goldengate_normal_1000.0.csv};
	\addplot [mark=square*, red, mark size=2] table [x=insertions, y=observed, col sep=comma] {./data/innodb/randomball_normal_1000.0.csv};
\end{axis}
\end{tikzpicture}
\label{fig:innodbnorm}
}\hfill
\subfloat[Buffer size and recycle rate]{
\tikzsetnextfilename{innodb-buffer-size}
\begin{tikzpicture}%[trim axis left, trim axis right]
\begin{axis}[
	%title=Uniform Key Distribution,
	width=0.48\textwidth,
	height=0.31\textheight,
	xlabel=Buffer size in MiB,
	xlabel near ticks,
	ylabel near ticks,
	ylabel style={align=center, text width=0.27\textheight},
	ylabel=Recycle rate (last million insertions),
	xmin=0,
	xmax=128,
	ymin=0,
	]
	\addplot [mark=*, Plum, mark size=2] table [x=Buffer, y=Rate, col sep=comma] {./data/innodb_buffer_size.csv};
\end{axis}
\end{tikzpicture}
\label{fig:innodbbuffer}
}

\caption{\innodb{} Insertion buffer recycle rates for various key distributions and memory sizes. (Higher is better)}}
\end{figure*}



%%% Local Variables:
%%% mode: latex
%%% TeX-master: "paper.tex"
%%% End:
