% !TEX root =  paper.tex

\secput{nonuniform}{Random Ball is Optimal}

In this section we prove \Cref{thm:random-opt}, showing that \RB{} is
$\Theta(1)$-optimal. 

\subsecput{nonuni-outline}{Outline of Proof}
We prove \Cref{thm:random-opt} in the following steps:
\begin{enumerate}
	\item No recycling strategy has a recycling rate exceeding
		$(2m+n-1)/\halfnorm{\p}$. (\secref{upper-statement})
	\item \RB{} matches that bound when $m \geq n$, leading to case (1) of
		\Cref{thm:random-opt}. (\secref{randomballwithmanyballs})
	\item There is an $\Theta(1)$-optimal strategy, \AE{}, when $m < n$. The
		recycling rate of \AE matches case (2) of \Cref{thm:random-opt}.
		(\secref{randomballaggroempty})
	\item By comparison to \AE{}, \RB{} is $\Theta(1)$-optimal when $m < n$.
		(\secref{rboptproof})
\end{enumerate}

\subsecput{upper-statement}{The Upper Bound}
We begin by proving an important lemma that will be used throughout, which we
refer to as the \defn{flow equation}. Then we proceed to prove the upper bound.

Let $A$ be a stateless strategy with stationary distribution $\chi^{A,\p}$. Let
$\phi_i$ be the event that $A$ picks bin $i$ to recycle. Let $\mathcal{R}_i^A =
E[R^A(\chi^{A,\p}) | \phi_i]$ be the number of balls recycled given that the
strategy picks bin $i$, and $f_i = P(\phi_j)$, the probability of picking the
bin $i$ in the stationary distribution. We note that $\mathcal{R}_i^A$ and
$f_i$ could alternatively be defined as limits of repeated applications of $A$
to any given starting state.

For a given bin $i$, we can analyze the ``flow'' of balls into and out of $i$.
When $k$ balls are thrown, $\p_ik$ of them are expected to land in $i$. For a
ball to leave $i$, $i$ must first be picked to be emptied by $A$, at which
point every ball in $i$ will be evicted. In the stationary distribution, the
net flow must be zero.  We can generalize to any set of bins and get:

\begin{lemma} \label{lem:gen-flow-equation}
	Let $A$ be a statelss strategy for a ball-recycling game with $n$ bins with
	probabilities $\p$. If $L$ is a subset of the bins, $\p_L = \sum_{\ell \in
	L} \p_\ell$, $f_L = \sum_{\ell \in L} f_\ell$ and $\mathcal{R}_L^A$ the
	conditional recycle rate given $A$ picks a bin in $L$, then
	\begin{equation}
		\p_L\mathcal{R}^A = f_L\mathcal{R}_L^A.
	\end{equation}
\end{lemma}

We will mostly use the following special case of the
\Cref{lem:gen-flow-equation}: 

\begin{lemma}[The Flow Equation] \label{lem:flow-equation}
	Let $A$ be a statless recycling strategy for a ball-recycling game with $n$
	bins with probabilities $\p$.  Then, for all $0\leq i < n$,
	\begin{equation}
		\p_i\mathcal{R}^A = f_i\mathcal{R}_i^A.
	\end{equation}
\end{lemma}

We now describe the main upper bound on the recycling rate of any recycling
strategy. In order to understand the intuition behind \Cref{lem:freqbound},
consider a given bin $i$. Intuitively, it makes sense to think that for a
reasonable recycling strategy the recycling rate of the other bins in the
system will go down as the number of balls $X_i$ in bin $i$ grows.  After all,
the $X_i$ balls in bin $i$ aren't available for recycling, until bin $i$ is
selected. If we assume this intuition as fact for the moment, this suggests
that the expected number of balls in bin $i$ should be greater than half the
recycling rate of bin $i$, perhaps excluding the last ball to land in the bin.

By the Flow Equation, this would suggest that
\[ \E{X_i} \geq \frac{1}{2}(\mathcal{R}_i^A - 1) =
\frac{1}{2}\left(\frac{\p_i}{f_i}\mathcal{R}^A - 1\right), \]
so that after summing over $i$, we obtain \Cref{lem:freqbound}.

However, the following strategy does not satisfy this assumption: for a given
bin $i$, have the strategy just pick the least full non-empty bins until $i$
has a few balls, then pick the fullest ones, then pick $i$ and repeat. Showing
that better strategies do not do this is non-trivial, and we prove
\Cref{lem:freqbound} by different means.

\begin{lemma}\label{lem:freqbound}
	Consider a ball-recycling game with $m$ balls, $n$ bins and distribution
	$\p$. If $A$ is a stateless strategy that picks bin $i$ with frequency
	$f_i$, then its recycle rate is bounded by
	\begin{equation}
		\mathcal{R}^A \leq \frac{2m+n-1}{\sum_j\frac{\p_j}{f_j}}.
	\end{equation}
\end{lemma}

Given a strategy $A$, the idea of the proof is to use the invariance of the
statistic
\begin{equation}
	Z(X) = \sum_{j = 1}^n \frac{X_j^2}{\p_j},
\end{equation}
under the action of $A$ on its stationary distribution. The application of $A$
to $Z$ together with the flow equation creates a factor of $\sum_j
\mathcal{R}_j$, which when solved for proves the bound. First, we begin with
some foundational lemmas, and then proceed to prove the main results.

\begin{lemma}
	Suppose $k$ balls are thrown into $n$ bins i.i.d.\ according to distribution
	$\mathbf{p}$. Let $B(j,k)$ be the binomial random variable denoting how
	many balls land in the $j$th bin.  The following hold:
	\begin{gather}
		\E{B(j,k)} = \p_jk \\
		\E{(B(j,k))^2} = \p_j(1-\p_j)k + \p_j^2k^2
	\end{gather}
\end{lemma}

\begin{proof}
	$B(j,k)$ is a binomial random variable with parameters $\p_j$ and $k$.
\end{proof}

Next, given a state $X$, we compute the effect of recycling the $\ell$th bin on
the $j$th component of $Z$. Note that if $A$ recycles bin $\ell$ of state $X$,
then $X_\ell = R^A(X)$.
\begin{lemma}
	In a ball-recycling game with $m$ balls, $n$ bins and probability
	distribution $\mathbf{p}$, if a strategy $A$ recycles bin $\ell$ in state
	$X$, then for $j \neq \ell$,
	\begin{multline}
		\E{(AX)_j^2}=X_j^2 + 2X_j\p_jR^A(X) \\
		+ \p_j(1-\p_j)R^A(X) + \p_j^2R^A(X)^2,
	\end{multline}
	where $R^A(X) = X_\ell$ is the number of balls recycled.
\end{lemma}

\begin{proof}
	\begin{align*}
		\E{(AX)_j^2} &= \E{(X_j+B(j,R^A(X)))^2} \\[1ex]
					 &= \begin{multlined}[t] X_j^2 + 2X_j\E{B(j,R^A(X))}\\
						 +\E{B(j,R^A(X))^2} \end{multlined} \\[1ex]
					 &= \begin{multlined}[t] X_j^2 + 2X_j\p_jR^A(X)\\+\p_j(1-\p_j)R^A(X)
						 + \p_j^2\left(R^A(X)\right)^2 \end{multlined}
	\end{align*}
\end{proof}

We now can use this result to compute the result of applying $A$ to $Z$. 
\begin{lemma}
	In a ball-recycling game with $m$ balls, $n$ bins and probability
	distribution $\mathbf{p}$, if a strategy $A$ recycles bin $\ell$ in state
	$X$, then
	\begin{multline}
		\E{Z(AX)}=Z(X) - \left(1+\frac{1}{\p_\ell}\right)\left(R^A(X)\right)^2 \\
		+ (2m+n-1)R^A(X)
	\end{multline}
\end{lemma}

\begin{proof}
	\begin{align*}
		\E{Z(AX)} 
		&= \sum_{j=1}^n \E{(AX)_j^2/\p_j} \\
		&= \frac{\E{(AX)_\ell^2}}{\p_\ell} + \sum_{j \neq \ell} \frac{\E{(AX)_j^2}}{\p_j} \\
		&= \begin{multlined}[t][6.25cm](1-\p_\ell)R^A(X) + \p_\ell\left(R^A(X)\right)^2 \\[0.75ex]
			\vspace*{-1.5ex}\shoveleft[0.25cm]{+ \sum_{j \neq \ell} \left( X_j^2/\p_j + 2X_jR^A(X)} \vphantom{\left(R^A(X)\right)^2}\right.\\
			\left.+ (1-\p_j)R^A(X) + \p_j\left(R^A(X)\right)^2 \right) \end{multlined} \\
		&= Z(X) - \left(2 + \frac{1}{\p_\ell}\right)\left(R^A(X)\right)^2 + 2mR^A(X) \\
		&\qquad+ \sum_j \left((1-\p_j)R^A(X) + \p_j\left(R^A(X)\right)^2\right) \\
		&= \begin{multlined}[t]Z(X) - \left(1+\frac{1}{\p_\ell}\right)\left(R^A(X)\right)^2 \\
			+ (2m+n-1)R^A(X) \end{multlined}
	\end{align*}
\end{proof}

Now, we can prove \Cref{lem:freqbound}.

\begin{proof}[Proof of \Cref{lem:freqbound}]
	Let $\chi^A$ be the stationary distribution relative to $A$. Let $\phi_j$
	be the event that $A$ recycles the $j$th bin of $\chi^A$, $R_j^A$ the
	random variable of how many balls are recycled given the $j$th bin is
	chosen by $A$, $\mathcal{R}_j^A=\E{R_j^A}$ and $f_j$ the probability that
	$A$ recycles that bin.  Because $\chi^A = A\chi^A$ by definition, we must
	have $\E{Z(A\chi^A)} = \E{Z(\chi^A)}$. Therefore:

	\begin{align*}
	\E{Z(\chi^A)} &= \E{Z(A\chi^A)}\\
	  			  &= \sum_j f_j \E{Z(A\chi^A)|\phi_j} \\
				  &= \begin{multlined}[t][6.5cm]\sum_j f_j \left(\E{Z(\chi^A)|\phi_j}
					  \vphantom{\left(1+\frac{1}{\p_j}\right)} \right. \\
					  - \left(1+\frac{1}{\p_j}\right)\E{\left(R_j^A\right)^2} \\
					  \left. \vphantom{\left(1+\frac{1}{\p_j}\right)} + (2m+n-1)\mathcal{R}_j^A\right) \end{multlined} \\
				  &\leq \begin{multlined}[t][6.5cm]\E{Z(\chi^A)} + (2m+n-1)\mathcal{R}^A \\
					  - \sum_j f_j\left(1 +\frac{1}{\p_j}\right)\left(\mathcal{R}_j^A\right)^2 \end{multlined} \\
				  &= \begin{multlined}[t][6.5cm]\E{Z(\chi^A)} + (2m+n-1)\mathcal{R}^A \\
					  - \sum_j \frac{1}{f_j}\left(\p_j^2 +\p_j\right)\left(\mathcal{R}^A\right)^2, \end{multlined}
	\end{align*}
	where the inequality is due to the Cauchy-Schwartz Inequality and the last
	line is because of the Flow Equation.
	
	Thus we have:
	\begin{align*}
		\mathcal{R}^A &\leq \frac{2m+n-1}{\sum_j \frac{1}{f_j}\left(\p_j^2 +\p_j\right)} \\
					&\leq \frac{2m+n-1}{\sum_j \frac{\p_j}{f_j}}
	\end{align*}
\end{proof}

\Cref{lem:freqbound} applies to the optimal deterministic strategy $\opt$
promised by \Cref{cor:opt-unique}, and we know that $\mathcal{R}^A\leq
\mathcal{R}^\opt$ for \textit{any} recycling strategy $A$.  Thus, by maximizing
the RHS of \Cref{lem:freqbound}, we can get an upper bound on the recycling
rate of any recycling strategy.

\begin{lemma}\label{lem:upperbound}
	Consider a ball-recycling game with $m$ balls, $n$ bins and distribution
	$\p$. For any recycling strategy $A$, 
	\[\mathcal{R}^A \leq \frac{2m+n-1}{\halfnorm{\p}}.\]
\end{lemma}

\begin{proof}
	This follows immediately from the Cauchy-Schwartz Inequality.
\end{proof}

\subsecput{randomballwithmanyballs}{\RB with \texorpdfstring{$m \geq n$}{m >= n}}

We show the following lower bound, which with \Cref{lem:upperbound}, shows
optimality when $m = \Omega(n)$.

\begin{lemma}\label{lem:random-ball-lower}
	\RB recycles at least $\frac{m}{\halfnormp}$ balls per round in
	expectation. 
\end{lemma}

\begin{proof}
	Let $\chi^\mathrm{RB}=(\chi_i^\mathrm{RB})$ be the random variable of the
	number of balls in each bin in the stationary distribution of \RB.  \RB
	recycles bin $i$ with probability $\frac{\chi_i^\mathrm{RB}}{m}$, and
	therefore the expected number of balls recycled from bin $i$ per round is
	$\frac{\E{\left(\chi_i^\mathrm{RB}\right)^2}}{m}$. The number of balls that
	land in bin $i$ per round is
	$\p_i\sum_{j=1}^n\frac{\E{\left(\chi_j^\mathrm{RB}\right)^2}}{m}$.  Since
	$X$ is distributed stationarily, we must have

	\begin{equation*}
		\p_i\sum_{j=1}^n\frac{\E{\left(\chi_j^\mathrm{RB}\right)^2}}{m}
	= \frac{\E{\left(\chi_i^\mathrm{RB}\right)^2}}{m} \geq \frac{\E{\chi_i^\mathrm{RB}}^2}{m},
	\end{equation*}

	using Jensen's Inequality. Clearing denominators, taking square roots and
	summing across $i$, we have
	\begin{align*}
		\left(\sum_{j=1}^n \E{\left(\chi_j^\mathrm{RB}\right)^2}\right)^{\hspace*{-0.5ex}\frac{1}{2}} \hspace*{-0.5ex}\sum_{i=1}^n \p_i^\frac{1}{2}
		&= \sum_{i=1}^n \left( \p_i\sum_{j=1}^n {\E{\left(\chi_j^\mathrm{RB}\right)^2}} \right )^{\hspace*{-0.5ex}\frac{1}{2}} \\
		&\geq \sum_{i=1}^n \E{\chi_i^\mathrm{RB}} = m.
	\end{align*}
	Therefore the expected recycle rate is
	\[ \sum_{j=1}^n \frac{\E{\left(\chi_j^\mathrm{RB}\right)^2}}{m}
	\geq \frac{m}{\left(\sum_{i=1}^n \sqrt{\p_i}\right)^2}
	= \frac{m}{\halfnormp}. \]
\end{proof}

\begin{corollary}\label{cor:rboptmanyballs}
	Consider a ball-recycling game with $m$ balls and $n$ bins.  If $m =
	\Omega(n)$, then \RB is asymptotically optimal among recycling strategies. 
\end{corollary}

\subsecput{randomballaggroempty}{\AE is Optimal}

In this section, we investigate \AE strategies, which aggressively recycle
balls outside a given subset of bins.  An \AE strategy runs one strategy on a
fixed subset of bins, but always chooses to recycle a bin outside of this set
if there exists one which has any balls.  Specifically, we show that a
$\Theta(1)$-optimal strategy on a particular $O(m)$ subset of the bins can be
extended to an $\Theta(1)$-optimal strategy on the full ball-recycling game by
aggressively emptying the
rest.

Consider a ball-recycling game with $m$ balls, $n$ bins, and ball distribution
$\p$. Let $L$ be some subset of bins and $S$ be a strategy on the \defn{induced
ball-recycling game} of $L$, which is the ball-recycling game with $m$ balls,
$|L|$ bins, and ball distribution $\q$, where 
\begin{equation*}
	\q_i = \frac{\p_i}{\sum_{\ell \in L} \p_\ell}.
\end{equation*}
Therefore, $\q$ is $\p$'s conditional probability distribution on $L$. We
define $L,S$-\AE to be the strategy which empties the lowest weight non-empty
bin in the complement of $L$ if one exists and otherwise performs $S$ on $L$.
Note that all the balls will be in $L$ whenever $S$ is performed, so this is
well-defined.

We begin by showing that there exists an $L$ and $S$ such that $|L|=O(m)$, $L$
contains all bins with weight at least $\frac{1}{m}$, and $L,S$-\AE is
asymptotically optimal. Note that when $m = \Omega(n)$, this is trivial,
because we can take $L$ to be all the bins and $S$ to be a $\Theta(1)$-optimal
strategy; however, this section provides stronger bounds when $m = o(n)$.
Intuitively, the idea is that very low weight bins won't be able to effectively
accumulate balls, so strategies do better to recover any balls in them than to
wait for more balls to land there.

\begin{lemma}\label{lem:aggroopt}
	There exists an $L$ and $S$ such that $|L|=O(m)$, $L$ contains all bins of
	weight at least $\frac{1}{m}$ and $L,S$-\AE is asymptotically optimal.
\end{lemma}

\begin{proof}
	By \Cref{lem:deterministic-opt}, there exists an optimal deterministic
	strategy $\opt$. Using the flow equation, \Cref{lem:freqbound} can be
	rewritten as:
	\[ \sum_{i=1}^n \mathcal{R}_i^\opt \leq 2m + n. \]
	Because $\opt$ will never recycle an empty bin, each $\mathcal{R}_i^\opt
	\geq 1$.  Therefore, there can be at most $m$ bins with average recycle
	rates at least 3. Let $L$ be this set of bins, together with any bins of
	weight at least $\frac{1}{m}$, and we will construct a strategy $S$ that
	aggressively empties the remaining bins into $L$.
	
	$S$ aggressively empties the complement of $L$, but also keeps a virtual
	configuration of where $\opt$ thinks the balls are, as well as a log of
	where $S$ has moved them. So when $S$ aggressively empties a bin, it also
	updates the log of each ball it throws, indicating where it landed. When
	$L^c$ is empty, it asks $\opt$ which bin to recycle based on the virtual
	configuration. If it says to recycle a bin in $L^c$, we use the logs to
	update where those balls will land in the virtual configuration. If it says
	to recycle a bin in $L$, we recycle those balls that are there in the
	virtual configuration, and leaving any others behind in that same bin.
	Thus $S$ performs $\opt$ but rushes ahead to recycle those balls outside of
	$L$.

	Now, consider $t$ rounds of $\opt$. For large enough $t$, $\opt$ will
	recycle on average at most 3 balls at a time from $L^c$. $S$ recycles at
	least 1 ball at a time from $L^c$ and exactly as many balls at a time from
	$L$.  Therefore for large $t$, $t$ rounds of $\opt$ will correspond to at
	most $3t$ rounds of $S$, and during this period $S$ will recycle the same
	number of balls. Thus $S$ is $1/3$-optimal.
\end{proof}

Next we compute the recycle rate of $L,S$-\AE as a function of the recycle rate
of $S$ on the induced ball-recycling game on $L$.

\begin{lemma}\label{lem:aggro-empty-bounds}
	If $\mathcal{R}^S$ is the recycle rate of $S$ (on $L$), and $q$ is the
	probability of a ball landing in $L^c$, then the recycle rate of $L,S$-\AE
	is
	\[\mathcal{R}^\textrm{AE} = \Theta\left(\frac{1}{(1 - q)/\mathcal{R}^S + q}\right)\]
\end{lemma}

\begin{proof}
	Consider a collection of recycling rounds of $L,S$-\AE where $t$ of those
	times $L,S$-\AE recycles a bin from $L$. Say $b$ balls are thrown from bins
	in $L$ and $a$ balls land in $L^c$. Now, if $m$ balls are thrown into bins
	of size at most $\frac{1}{m}$, then the expected number of empty bins is at
	most \[ m \left(1 - \frac{1}{m}\right)^m \leq \frac{m}{e}.\] Because fewer
	thrown balls will have fewer collisions, this means the expected number of
	non-empty bins when $k \leq m$ balls are thrown into $L^c$ is at least
	$\left(1-\frac{1}{e}\right)k$, requiring at least as many time steps to
	aggressively empty. Thus, for large $t$, the expected number of turns
	required to empty the $a$ balls out of $L^c$ is at least
	$\left(1-\frac{1}{e}\right)\frac{a}{1-q}$. Whereas even if the balls were
	recycled from $L^c$ one at a time this expected number of turns is at most
	$\frac{a}{1 - q}$ turns. The number of balls recycled during this period is
	$b + \frac{a}{1 - q}$, and we have shown the number of rounds $\rho$
	satisfies: \[\rho = \Theta\left(t + \frac{a}{1-q}\right).\]

	For large enough $t$, $b = \Theta\left(t\mathcal{R}^S\right)$ and
	$a = \Theta\left(tq\mathcal{R}^S\right)$, so the overall recycle rate
	$\mathcal{R}^\textrm{AE}$ therefore satisfies
	\begin{align*}
		\mathcal{R}^\textrm{AE} &= \Theta\left(\frac{t\mathcal{R}^S + tq\mathcal{R}^S/(1 - q)}{t + tq\mathcal{R}^S/(1 - q)}\right) \\
		&= \Theta\left(\frac{1}{(1 - q)/\mathcal{R}^S + q}\right)
	\end{align*}
\end{proof}

\subsecput{rboptproof}{\RB is Optimal}

In this section we will further examine the performance of \RB and show that it
is asymptotically optimal. We first describe a sufficient condition for
optimality of a strategy based on its recycle rate on $L$, then show that \RB
satisfies this criterion.

\begin{lemma}\label{lem:rl-opt}
	Let $L$ be a set of $O(m)$ bins for which there exists a strategy $T$ such
	that $L,T$-\AE is asymptotically optimal.  Let $\mathcal{R}^{\opt_L}$ be
	the recycle rate of the optimal strategy on the induced ball-recycling game
	of $L$.  For a given strategy $S$, let $\mathcal{R}_L^S$ be the conditional
	recycle rate of $S$ in the stationary distribution given that a ball in $L$
	is selected, and $q$ be the probability that a ball lands in $L^c$, i.e. $q
	= \sum_{k \in L^c} \p_k$. If either
	\[ \E{\mathcal{R}_L^S} = \Omega(\mathcal{R}^{\opt_L})
	\quad\textrm{or}\quad\E{\mathcal{R}_L^S} =
	\Omega\left(\frac{1}{q}\right),\]
	then $S$ is asymptotically optimal.
\end{lemma}

\begin{proof}
	By applying \Cref{lem:gen-flow-equation}, the subset variant of the
	flow equation, to $L$, 
	\[ f_L \mathcal{R}_L^S = (1-q)(f_L \mathcal{R}_L^S + (1-f_L) \mathcal{R}_{L^c}^S), \]
	where $f_L$ is the stationary probability of $S$ picking a bin in $L$.
	Solving for $f_L$,
	\begin{align} 
		f_L = \frac{\mathcal{R}_{L^c}^S}{q\mathcal{R}_L^S +
          \mathcal{R}_{L^c}^S}. \label{eqn:fL}
	\end{align}

	Suppose $\mathcal{R}_L^S = o\left(\frac{1}{q}\right)$ and $\mathcal{R}_L^S$
	is $\Theta(1)$-optimal on $L$. If $\mathcal{R}_L^S\leq \frac{1}{q}$, then
	$f_L \geq \frac{1}{2}$, and so because $\mathcal{R}^S = f_L\mathcal{R}_L^S
	+ (1-f_L)\mathcal{R}_{L^c}^S$, we must have $\mathcal{R}^S =
	\Omega(\mathcal{R}_L^S)$.

	Now, using \Cref{lem:aggroopt}, let $L$ and $T$ be such that $L,T$-\AE is
	asymptotically optimal, and let $\mathcal{R}^\textrm{AE}$ be its expected
	recycle rate. By \Cref{lem:aggro-empty-bounds},
	\begin{align*} 
		\mathcal{R}^\mathrm{AE} &= \Theta\left( \frac{1}{(1-q)/\mathcal{R}^T + q}\right) \\
		&= O\left(\mathcal{R}^T\right) = O\left(\mathcal{R}_L^S\right) = O\left(\mathcal{R}^S\right),\label{eqn:RAE}
	\end{align*} 
	so $S$ must be asymptotically optimal.

	If $\mathcal{R}_L^S = \Omega\left(\frac{1}{q}\right)$, then
	$\mathcal{R}_L^S > \frac{\alpha}{q}$ for some $\alpha$. Rearranging
	Equation~\eqref{eqn:fL} and multiplying by
	$\mathcal{R}_L^S$ yields
	\[ f_L \mathcal{R}_L^S = \frac{1}{\frac{q}{\mathcal{R}_{L^c}^S} + \frac{1}{\mathcal{R}_L^S}}. \]
	Here $\frac{1}{\mathcal{R}_L^S} \leq \frac{q}{\mathcal{R}_{L^c}^S}$, so
	$f_L\mathcal{R}_L^S = \Omega\left(\frac{1}{q}\right)$, and thus
	$\mathcal{R}^S = \Omega\left(\frac{1}{q}\right)$ as well. Now we can
	compare to $L,T$-\AE as above:
	\[ 	\mathcal{R}^\mathrm{AE}
		= \Theta\left( \frac{1}{(1-q)/\mathcal{R}^T + q}\right)
		= O\left(\frac{1}{q}\right)
		= O\left(\mathcal{R}^S\right)
		\]
	so in this case $S$ is asymptotically optimal as well.
\end{proof}	

We can now prove \Cref{thm:random-opt}.

\begin{proof}[Proof of \Cref{thm:random-opt}]
	If $m=\Omega(n)$, then by \Cref{lem:upperbound,lem:random-ball-lower} we are done.

	Otherwise, let $L$ be a set of $O(m)$ bins for which there exists a
	strategy $T$ such that $L,T$-\AE is asymptotically optimal. We will prove
	the result for a slightly modified \RB that only recycles 1 ball outside of
	$L$ even if more are available; that is, it moves only one of the balls in
	the bin. Since this strategy is worse than \RB, this will be sufficient. We
	number the bins so that the first $|L|$ bins comprise $L$.
	
	If $\mathcal{R}_L \geq \frac{1-q}{q}$, then we are done by
	\Cref{lem:rl-opt}. Otherwise, in the stationary distribution, when a
	bin in $L$ is recycled, the expected number of balls which land in $L^c$ is
	$q\mathcal{R}_L < 1-q$. When a bin in $L^c$ is recycled, the expected
	number of balls which land in $L$ is $1-q$. Thus \RB must pick a bin in $L$
	more than half the time, and so the expected number of balls in $L$ must be
	more than $\frac{m}{2}$.

	Now analogously to the proof of \Cref{lem:random-ball-lower}, we have:
	\begin{align*}
		\p_i\left(\sum_{j=1}^{|L|} \E{\left(\chi_j^\mathrm{RB}\right)^2} + \sum_{j=|L|+1}^n \E{\chi_j^\mathrm{RB}}\right) &= \E{\left(\chi_i^\mathrm{RB}\right)^2} \\
		&\geq \E{\chi_i^\mathrm{RB}}^2.
	\end{align*}

	Thus,
	\[ \E{\chi_i^\mathrm{RB}} \leq \sqrt{\p_i}\left(\sum_{j=1}^{|L|} \E{\left(\chi_j^\mathrm{RB}\right)^2} + \frac{m}{2}\right)^{\frac{1}{2}}.\]
	Summing over $i \leq |L|$ yields
	\[ \E{\chi_L^\mathrm{RB}} \leq \left(\sum_{i=1}^{|L|} \sqrt{\p_i}\right)\left(\sum_{j=1}^{|L|} \E{\left(\chi_j^\mathrm{RB}\right)^2} + \frac{m}{2}\right)^\frac{1}{2}, \]
	where $\chi_L^\mathrm{RB}$ is the expected number of balls in $L$. Now,
	\begin{align*}
		\mathcal{R}_L^\mathrm{RB} &\geq \frac{1}{m}\sum_{j=1}^{|L|} \E{\left(\chi_j^\mathrm{RB}\right)^2} \\
								  &\geq \frac{\E{\chi_L^\mathrm{RB}}^2}{m\left(\sum_{i=1}^{|L|} \sqrt{\p_i}\right)^2} - \frac{1}{2} \\
								  &> \frac{m}{4\halfnorm{\mathbf{p}_L}} - \frac{1}{2},
	\end{align*}
	where $\p_L$ is the conditional probability distribution on $L$ obtained
	from $\mathbf{p}$. The last inequality holds because there are at least
	$\frac{m}{2}$ balls in $L$ in expectation.

	Thus by \Cref{lem:upperbound}, \RB is asymptotically optimal on the induced
	system of $L$, and therefore \RB is asymptotically optimal by
	\Cref{lem:rl-opt}.
\end{proof}




%%% Local Variables:
%%% mode: latex
%%% TeX-master: "paper.tex"
%%% End:
