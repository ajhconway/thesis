\section{Related Work}\label{sec:fsa-related}
%AC: old text at bottom of tex file
Prior work on file system aging falls into three categories:
techniques for artificially inducing aging, for
measuring aging, and for mitigating aging.
% In this section we first survey some of the tools that have been developed to
% age a file system synthetically, along with measures for quantifying aging.
%Then we discuss some of the techniques that file systems currently include to
%mitigate aging.
%Most file systems also include techniques to mitigate aging.  Here we review
%the main approaches that file systems take.

% \fixmerob{The big problem with the related work is that it is simply a
%   list of prior work with no comparisons/contrasts or wisdom.  At the
%   end of the related work section, the reader should be able to place
%   this work in the context of past work, and be able to describe the
%   research hole that this paper is filling.  I'm not getting that from
%   the current related work write-up.  Here are questions to answer in
%   this section: Why are we doing this research if Smith and Seltzer
%   already did aging?  Why do we use a dynamic score instead of their
%   static score?  What about distributions of what-not affect aging?
%   The sentence seems completely disconnected from the topic of this
%   paper.  Why aren't we using Smith and Seltzer's traces to age our
%   file systems?  What is deficient about the artificial aging tools of
%   the past?}



\subsection{Creating Aged File Systems}

The seminal work of Smith and Seltzer~\cite{SmithSe97} created a
methodology for 
simulating and measuring aging on a file system---leading
to more representative benchmark results than 
running on a new, empty file system.
The study is based on data  collected
from daily snapshots of over fifty real file systems from five servers
over durations ranging from one to three years. 
An overarching goal of Smith and Seltzer's work was to evaluate
file systems with representative levels of aging.
%our goal is instead to identify the root causes of aging
%and evaluate preventive techniques under more extreme circumstances.
%Our paper extends Smith and Seltzer's layout score to capture
%metadata and dynamic access patterns, and 
%identifies benchmarks that induce precipitous aging on most
%file systems.

%Their method takes a long time
%to construct workloads while fails to capture the directory hierarchy of
%the file system. 

%\fixmemab{see comment}
%\fixmemab{I don't like this second sentence. It
%  sounds insinuatingly critical. Better to say what they do and say
%  how this compares with what we do. }

Other tools have been subsequently developed for
synthetically aging a file system.
In order to measure NFS performance,
TBBT~\cite{ZhuChCh05} was designed 
to synthetically age a disk 
to create a initial state for NFS trace replay.
%% initial \fixmemfc{what does ``initial'' mean in this context?},
%% synthetically aged state upon which to replay an NFS trace.  If a file
%% system snapshot is copied onto a fresh file system, the placement will
%% be essentially optimal \fixmerob{Is that really true?}  \fixmemfc{It
%%   is not.  I'd kill it but I don't understand what it means in this
%%   context.}.
%% In order to create a more\mfc{more than what?}
%% realistic initial state for trace replay,
%\mfc{IF we are cutting, this sentence can go: TBBT first creates a
%namespace hierarchy, then interleaves synthetic operations so that
%allocations are more fragmented.}

The Impressions framework~\cite{AgrawalArAr09} was designed so that
users can synthetically age a 
file system by setting a small number of parameters,
%% Impressions~\cite{AgrawalArAr09} is a framework for creating and
%% compactly representing synthetic file system images.  Impressions
%% exports\mfc{``exports''?} a number of parameters that influence aging,
such as the organization of the directory hierarchy.
Impressions also lets users specify a target layout score for the resulting image.
%layout score of the resulting image \fixmerob{``export''?}.
%\fixme{BILL: identifying which dimensions contribute to aging
%  may be file system specific, but applicaiton (trace) replays
%  are always file system agnostic}

Both TBBT and Impressions
create file systems with a specific level of fragmentation,
%focus on creating an aged file system with a
%specific level of fragmentation, while
whereas our study identifies realistic
workloads that induce fragmentation. 
%to see if they lead to
%\mfc{Is this accurate?}

\subsection{Measuring Aged File Systems}


Smith and Seltzer also introduced a \defn{layout score} for studying aging,
which was used by subsequent studies~\cite{ahn02mascots,AgrawalArAr09}. Their
layout score is the fraction of file blocks that are placed in consecutive
physical locations on the disk. We introduce a variation of this measure, the
\defn{dynamic layout score} in \secref{metrics}.

%\mfc{Don't we explain this in detail
%  later?  If so, then we can shorten this to a forward reference, to a
%  particular section.  And thus cut lines from the paper:
%In this paper, we use a \emph{dynamic} version
%of this layout score that performs a recursive scan through the file system and
%tracks what fraction of blocks are requested from the driver in sequential
%order. Thus this dynamic layout score both incorporates metadata traversal and
%allows for the OS to perform some collalescing of requests. }

The \defn{degree of fragmentation} (\defn{DoF}) is used in the study of
fragmentation in mobile devices~\cite{JiChSh16}. DoF is the ratio of the actual
number of extents, or ranges of contiguous physical blocks, to the ideal number
of extents.  Both the layout score and DoF measure how one file is fragmented.

\iffalse
\fixme{BILL: do any
  fragmentation tools report layout scores?  e2fsprogs?}\fixmeac{I
  checked the man pages quickly for e2fsprogs, and I don't think they
  do}\fixme{Yang: Just to mention it contains tools for reporting file
  fragmentation, so one can probably calculate from there. Another
  thing, there is one paper "The Effects of Filesystem Fragmentation"
  describing some theoretic fragmentation but mainly on video files
  (so they are big)}
\fi

Several studies have reported file system statistics
% that can be used to reason about aging
such as number of files, distributions of
file sizes and types, and organization of file system
namespaces~\cite{AgrawalBoDo07,Downey01,RoselliLoAn01}.
%\fixme{BILL: validate some of these and choose:
%  a five-year study of file system metadata,
%  structural cause of file size distributions,
%  a comparison of file system workloads,
%  a study of file sizes and functional lifetimes}
%These statistics can be used as parameters to aging frameworks like TBBT and Impressions~\cite{ZhuChCh05,AgrawalArAr09}.
These statistics can inform parameter choices in aging frameworks like TBBT and Impressions~\cite{ZhuChCh05,AgrawalArAr09}.
%For example, as we describe
%later on, the file-size distribution can have an impact on fragmentation.

%% Finally, the \texttt{genbackupdata}~\cite{genbackupdata} tool
%% synthetically generates file system images, but is primarily designed
%% for studying backup systems, not aging, and does not provide the same
%% degree of control over system behavior as Impressions.
%% \fixme{This generates a disk image in a principled way, but is not related to aging}


%% Local Variables:
%% mode: latex
%% End:
