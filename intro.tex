Dictionaries are fundamental data structures that map a set of keys to values.
A dictionary generally must support insertion and lookup, but also optionally
delete, update, successor and scan. A systems that implements a dictionary is
called a key-value store. Dictionaries lie at the heart of most storage systems
and also play an essential role in many algorithms.

The work presented here examines problems related to the theory and
implementation of high-performance external memory dictionaries.

Since the invention of the B-tree 50 years ago, the theory of external memory
dictionaries and the implementation of key-value stores has progressed to some
extent independently and in parallel. Currently, most external memory key-value
stores used in practice are based on log-structured merge trees (LSMs), which
are suboptimal from a theoretical standpoint.

From a theoretical perspective, there are many dictionary data structures which
are optimal under different models. In the comparison DAM model, where the only
operations permitted on key are key comparisons, the \bet was the first known
optimal dictionary, and it was followed by an LSM derivative, the COLA, and
several other data structures. In the broader DAM model, more general
operations---in particular hashing---are permitted, which improves the
performance of optimal data structures. In this model, the only previously
known optimal dictionary is the external memory hash table of Iacono and
P\v{a}tra\c{s}cu. Of these optimal data structures, only the \bet has been used
successfully in a handful of systems.

\section{Key Differences between Theory and Practice}

Although a proper exposition of this schism between the data structures used in
practice and those which are optimal in theory is beyond the scope of this
work, it is helpful to examine some of the reasons here in order to motivate
and contextualize the results.

\paragraph{Models}
One of the difficulties in translating theoretical results into systems is
understanding the strengths and limitations of the models used. The disk access
model (DAM) has long been the standard model for external memory algorithms,
but while it offers a reasonable approximation of the perforamnce of hard
drives, it can distort performance by a factor of 2 compared to more realistic
models, such as the affine model. These sorts of constant factors are
irrelevant in theory, but meaningful in practice.

The assumptions used also matter. For example, it is often the case that a
system may have a cache which is 25--50\% the size of the dataset, whereas many
theoretical results, such as the lower bounds alluded to above, assume that the
cache is much smaller, typically $M = o(N/\log{N})$. Thus, at first glance,
these results may not directly apply.

\paragraph{Hardware}
As technology changes and new storage hardware is developed, new models and
design principles are required to understand and leverage their performance. As
the premium storage technology has evolved from hard drive to solid state
device to block-addressable non-volatile memory, much of the external memory
literature has remained relatively constant. Interestingly enough, the
performance of SSDs and NVMe devices is perhaps best captured by a classical
model: the parallel disk access model (PDAM). As a result, in many settings,
variations of classical external memory data structures perform well on these
devices.

\paragraph{Filters}
A filter is a probabilistic set-membership data structure with one-sided error.
The first and most well-known example is the Bloom filter, but many other
variants have been proposed and implemented, such as the cuckoo filter and the
quotient filter. Filters are commonly used to optimize the lookup performance
of LSMs and under some conditions can reduce the number of IOs per point lookup
to 1.

Filters are commonly only used in memory (although cuckoo and quotient filters
have good performance characteristics in external memory as well), and across
the dataset consume at least $\Omega{N/\log{N}}$ space---implying that $M =
\Omega{N/\log{N}}$. This part of the parameter space is less theoretically
interesting, because the principle lower bounds do not hold. However, as a
result, the application of filters to common data structures which
asymptotically dominate LSMs, such as COLAs and \bets, has been overlooked, and
these data structures have received much less exposure to the systems
community.

Thus filters have steered systems both towards LSMs, by ``fixing'' their
lookups, and towards hardware configurations with a relatively large amount of
cache, to enable their use.

\section{This Work}

This work studies a collection of dictionary problems, each of which lies
somewhere between theory and practice. The themes above---models and their
limitations, changing hardware and the use of filters---frequently recur.
However, rather than dividing the theoretic and practical study of
dictionaries, these themes reflect the flow of ideas back and forth between
them. This yields interesting and surprising results, both where innovations
and ideas in systems have influenced theoretical data structures, and also
where those data structures form the foundation for new highly performant
systems.

\paragraph{Chapter 2: Optimal Hashing in External Memory}

This is evident in \textit{Optimal Hashing in External Memory}, in which
LSMs---commonly used in practice, but theoretically suboptimal---are modified
using a new type of filter, the routing filter, to create a simple optimal
external memory dictionary, the BOA. Pushing these ideas further yields the BOT,
which is optimal over larger parameter ranges, and the COBOT, which is the
first optimal cache-oblivious dictionary.

This content was presented at ICALP 2018 by myself, Mart\'in Farach Colton and
Phillip Shillane.

\paragraph{Chapter 3: SplinterDB}

This chapter introduces SplinterDB: a highly concurrent key-value store
designed to perform on NVMe. SplinterDB takes the ideas underlying BOAs and
BOTs, including routing filters, and implements them within a \bet. SplinterDB
outperforms RocksDB, a state-of-the-art key-value store, by 2--8$\times$ on
insertions and 1.2--2$\times$ on point lookups.

This content will appear at ATC 2020 and is by myself, Abhishek Gupta, Vijay
Chidambaram, Mart\'in Farach-Colton, Richard Spillane, Amy Tai and Rob Johnson.

\paragraph{Chapter 4: File System Aging}

File system aging is thought to be a solved problem: while older file systems
are known to age, modern file systems (and devices) are believed to only age
under adversarial workloads. However, models for block allocation suggest that
the approaches taken by most file systems should lead to aging under a broad
range of workloads.

In this work, we present an aging tool based on the version control system,
git, which replays development histories spanning years of use.  Using this
tool, we show that 5 popular file systems suffered catastrophic aging on hard
drives and substantially aging on SSDs.

On the other hand, BetrFS, a file system which uses \bets to manage its on-disk
data, should not age because it algorithmically moves stored data to maintain
locality. This is borne out using the git aging tool: BetrFS does not age at
all under the same workload.

This content was presented at FAST 2017 by myself, Ainesh Bakshi, Yizheng Jiao,
William Jannen, Yang Zhan, Jun Yuan, Michael A Bender, Rob Johnson, Bradley C
Kuszmaul, Donald E Porter and Mart\'in Farach-Colton.

\paragraph{Chapter 5: Optimal Ball Recycling}

A popular optimization technique for B-tree-based databases is the use of an
insertion buffer. This is an in-memory buffer, which stores insertions as they
arrive with the hope of batching them together before they are written to the
leaves of the B-tree. This chapter models and analyzes insertion buffers (as
well as the related update buffers) and provides a tight upper bound of the
performance improvement that can be obtained. This result undermines the common
assumption that large caches can be used to fix performance problems in storage
systems.

This content appeared at SODA 2018 by Michael Bender, Jake Christensen, myself,
Martin Farach-Colton, Rob Johnsonand Meng-Tsung Tsai.
