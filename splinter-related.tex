\section{Related Work}
\label{sec:related}

The closest work to ours is Tucana~\cite{DBLP:conf/usenix/PapagiannisSGB16}, a
\bet optimized for SSDs.  They also focus on CPU cost, concurrency, and write
amplification.  Our work pushes this to the even more demanding case of NVMe
devices.  

\textbf{Size-Tiering.}  Cassandra~\cite{cassandra}, Scylla~\cite{scylla}
PebblesDB~\cite{DBLP:conf/sosp/RajuKCA17}, and
RocksDB~\cite{facebook-rocksdb2018} (in ``universal compaction'' mode) use
size-tiering to reduce write amplification.  Size-tiering delays compaction of
sorted runs in order to reduce write amplification.  This can harm query
performance because queries must search in more runs to find the queried item.
Fluid LSMs~\cite{DBLP:conf/sigmod/DayanI18},
Dostoevsky~\cite{DBLP:conf/sigmod/DayanI18}, LSM
bushes~\cite{DBLP:conf/sigmod/DayanI19}, and
Wacky~\cite{DBLP:conf/sigmod/DayanI19} use hybrids between size-tiering and
level-tiering to tune the trade-off between write amplification and query
performance.  See~\cite{DBLP:conf/sosp/RajuKCA17} for a survey of
LSM-compaction schemes.

\textbf{Bloom filters in LSM trees.}  Almost all LSM trees use Bloom
filters~\cite{DBLP:journals/cacm/Bloom70} to improve point-query performance,
and specifically to mitigate the impact of size-tiering.
Monkey~\cite{DBLP:conf/sigmod/DayanAI17} and
ElasticBF~\cite{DBLP:conf/usenix/LiTGLX19} investigated how to allocate RAM to
Bloom filters to improve query performance.  Bloom filters do not affect range
query performance, however, since they support only point queries.

\textbf{Additional indexing.}  Numerous \kv stores use additional indexing to
improve query performance.  For example,
COLAs~\cite{DBLP:conf/spaa/BenderFFFKN07} use fractional cascading,
\bets~\cite{DBLP:conf/soda/BrodalF03} follow a B-tree-like structure, and
PebblesDB uses randomized skip-list-based fractional cascading (referred to as
guards in~\cite{DBLP:conf/sosp/RajuKCA17}).  External-memory skip lists were
analyzed in~\cite{DBLP:conf/pods/BenderBJKMPSSZ16:set} and write optimized
in~\cite{DBLP:conf/pods/BenderFJMMPX17}.  The main technical challenge in such
skip lists is the high variance of the size of runs, which in PebblesDB was
addressed by turning long runs into mini-B-trees.
%% Since our data structure is a based on a \bet, many of these schemes
%% are not as relevant here.

\textbf{Write amplification vs. range queries.}  Several systems sacrifice
range-query performance in order to reduce write amplification in other ways.
Wisckey~\cite{DBLP:journals/tos/LuPGAA17} reduces write amplification by
declustering their \kv store: they log values and only store keys in the
LSM-Tree.  Since values are stored on disk in arrival order, a range query must
gather values from the log.  On NVMe, this is not a problem once the values are
4KB or larger.  However, for smaller values, this can induce huge read
amplification, limiting range query performance to a tiny fraction of device
bandwidth.  HashKV~\cite{DBLP:conf/usenix/ChanLLX18} builds on Wisckey by
introducing hash-based data-grouping to further reduce write amplification, but
inherits Wisckey's range query performance limitations.

\alex{Going to have to fix this.} Other systems improve write amplification by
sacrificing range queries altogether.  Conway et
al.~\cite{DBLP:conf/icalp/ConwayFS18} describe a write-optimized hash table,
called the BOA, that also uses size-tiering with an LSM.  In a BOA, SSTables of
sorted runs are replaced with hash tables.  They also introduce the concept of
a routing filter, which extends the functionality of Bloom filters, in order to
speed up queries.  The principle advantage of routing filters is that
performance does not degrade as much when they don't fit in RAM.  The BOA meets
a provable lower bound on the I/O costs of insertions and
queries~\cite{DBLP:conf/soda/IaconoP12}.  Thus the BOA is essentially the best
possible on-disk data structure for random insertions and point queries.  The
downside is that the BOA does not support range queries, which are crucial to
many \kv-store applications.  LSM-tries~\cite{DBLP:conf/usenix/WuXSJ15}
organize the LSM tree using tries, resulting in reduced write amplification.
However, LSM-tries do not support range queries.

\textbf{Other approaches.}  Researchers have also attempted to reduce write
amplification by exploiting special hardware features such as flash translation
layers~\cite{DBLP:conf/usenix/MarmolSTR15} and vector
interfaces~\cite{DBLP:conf/cloud/VasudevanKA12}.
VT-Tree~\cite{DBLP:conf/fast/ShettySMASZ13} uses indirection to avoid copy data
that is already sorted.  ``Trivial moves'' are a similar idea is in RocksDB and
PebblesDB.  TRIAD~\cite{DBLP:conf/usenix/BalmauDGZYAGK17} reduces write
amplification by holding hot keys in memory, delaying compaction until
different runs have significant key overlap, and by reducing redundancy between
log and LSM tree writes.  All these techniques are orthogonal to our work and
can be used in conjunction with our techniques.

Concurrency is also an important aspect of key-value store performance. One of
the first works in increasing concurrency in LSM-based stores was
cLSM~\cite{DBLP:conf/eurosys/Golan-GuetaBHK15} which introduces a new
compaction algorithm.  Zuo et al.~\cite{DBLP:journals/tos/ZuoHW19} show how to
tune a cuckoo hash for NVM.  Such a scheme suffers from high write
amplification, since each insertion must re-write all keys in a data block.
Zuo et al.~do not report write amplification numbers but instead focus on
concurrency.

Recent work on fast \kv stores includes
GearDB~\cite{DBLP:conf/fast/YaoWHZLXH19}, a \kv store that avoids garbage
collection on HM-SMR.  Eisenman et
al.~\cite{DBLP:conf/eurosys/EisenmanGAADHPC18} address the issue of large DRAM
requirements of \kv stores for NVM via several techniques, including in-memory
compression and NVM-specific caching schemes.  Kourtis, et al.~describe several
systems-level optimizations for improving key-value-store throughput on NVMe,
such as efficient use of user-level asynchronous I/O and low-latency
scheduling~\cite{DBLP:conf/fast/KourtisIK19}.  Their techniques are largely
orthogonal to the work in this paper.  Kaiyrakhmet, et al.~use
persistent memory to simplify and improve performance relative to
LevelDB~\cite{DBLP:conf/fast/KaiyrakhmetLNNC19}.

