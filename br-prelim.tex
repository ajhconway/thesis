\section{Ball Recycling and Markov Theory}\label{sec:br-prelim}

We begin our analysis of ball-recycing games with some preliminary results.  In
particular, we show that all finite-state ball-recycling strategies have
stationary distributions.

\subsection{Ball-recycling games are Markov decision processes.}\label{sec:br-mdps}

This section makes use of the standard theory of Markov chains and Markov
decision processes; for an introduction see {\it e.g.}
Kallenberg~\cite{Kallenberg16}.

In a ball-recycling game, we represent the configuration of the balls as a
vector $X=(X_i)$ of length $n$, where $X_i$ is the number of balls in the $i$th
bin.  Since the number of balls is finite, there are only a finite number of bin
configurations.

A recycling strategy $A$ takes as input the current bin configuration $X$
together with an internal state $S$, and selects a non-empty bin to recycle;
the next state is obtained by removing all the balls from the selected bin and
re-throwing them according to $\p$. The bin selection may be randomized.  We
write $A^iX$ for the state obtained after $i$ rounds of recycling using
strategy $A$.  In each round, the recycling algorithm earns a reward equal to
the number of balls recycled in that round.

In this way, the ball-recycling game is a Markov decision process, and we are
interested in policies that maximize the expected average recycling rate,
defined for a policy $A$ as
\begin{equation*}
	\mathcal{R}^A = \lim_{T\rightarrow\infty} \frac{1}{T}\sum_{t=0}^{T} R(A^tX_0).
\end{equation*}

Note that Markov decision processes are very general.  For example, in a Markov
decision process the policy may vary over time, and may even take the entire
history of the process and its own past decisions into account when deciding on
its next action.  Thus, for some strategies $A$, the limit $\mathcal{R}^A$ may
not exist.  In the literature of Markov decision processes, this is often
handled by taking the $\limsup$ instead of the limit. However, for any Markov
decision process, any strategy that maximizes the limit also maximizes the
$\limsup$~\cite{Kallenberg16}.  Therefore, for simplicity, we will focus only
on strategies for which the limit is well-defined, and the results will
generally hold for the $\limsup$ of arbitrary strategies.

A Markov decision process policy is \defn{deterministic} if it decides on its
next action based solely on the current state, \textit{i.e.}\ without looking at
history, the number of time steps that have passed or by flipping random
coins.  A deterministic policy can be represented as a simple table mapping
each state to a single action to be taken whenever the system is in that state.

The specific strategies we analyze are \defn{finite-state} strategies, where
the internal state has only finitely many configurations.  When we prove our
lower bounds, we will further restrict ourselves to \defn{stateless}
strategies, where there is a unique internal state. In order to do so, we make
use of the following lemma.

\begin{lemma}[\cite{Kallenberg16}]\label{lem:deterministic-opt}
	There exists a stateless deterministic recycling strategy $\opt$ that
	achieves the optimal expected average recycling rate.
\end{lemma}
\begin{proof}
	This follows from Kallenberg's Corollary 5.4.
\end{proof}

\subsection{Stationary distributions of recycling strategies}\label{sec:br-markov}

A ball-recycling game and a recycling strategy together define a Markov
process on the state space; the space of pairs comprising a balls-and-bins
configuration and an internal state. If the strategy is stateless, this is a
Markov process on the balls and bins space.

There are a finite number of balls-and-bins configurations. Therefore, a
ball-recycling game with a finite-state recycling strategy defines a finite
Markov process, and so has at least one stationary distribution.

We now show that stateless recycling strategies result in Markov processes with
unique stationary distributions.  The following lemma shows that, when we look
only at the bin configurations, recycling games have properties analogous to
irreducibility and aperiodicity in Markov chains. 

\begin{lemma}\label{lem:ergodic}
	For any ball-recycling game with $m$ balls and $n$ bins there is an
	$\epsilon>0$ such that, for all bin configurations $X$ and $Y$, and for all
	recycling strategies, the probability that $X$ reaches $Y$ within
	$\min(m,n)$ steps is at least $\epsilon$.  Furthermore, every bin
	configuration can transition to itself in one time step.
\end{lemma}
\begin{proof}
	We just need to show a sequence of outcomes for ball tosses that transform
	$X$ into $Y$, no matter which bins the recycling strategy chooses to empty.
	So, imagine that, at each step, all the recycling balls land in occupied
	bins, so that at each step, the number of occupied bins goes down by 1.
	After at most $\min(m,n)-1$ steps, all the balls will be in a single bin.
	On the next round, the recycling strategy must choose that bin, causing all
	the balls to be rethrown.  There is some non-zero probability that they
	land in configuration $Y$.

	For the second observation, simply note that all the recycled balls may
	happen to land in the bin from whence they came.
\end{proof}

\begin{lemma} \label{lem:stationary}
	\leavevmode
	\begin{enumerate}
	\item All ball-recycling games using stateless recycling strategies have
		unique stationary distributions which are equal to their limiting
		distributions.
  \end{enumerate}
\end{lemma}
\begin{proof}
	Having fixed a stateless recycling strategy, the ball-recycling game is a
	Markov process on the balls-and-bins configuration. By
	\cref{lem:ergodic}, this process is irreducible and aperiodic, and so
	has a unique stationary distribution equal to its limiting distribution.
\end{proof}

Together with \cref{lem:deterministic-opt}, we have:
\begin{corollary} \label{cor:opt-unique}
	For any ball-recycling game, there exists an optimal strategy with a unique
	stationary distribution.
\end{corollary}

We now show that two of the three main recycling strategies studied in this
paper yield unique stationary distributions.  The strategies are:
\begin{itemize}
\item \FB: selects the bin that has the most balls;
\item \RB: selects a ball uniformly at random and recycles whichever bin it is in;
\item \GG: selects the bins in a round robin sequence.
\end{itemize}

\FB is a deterministic strategy, \RB is a stateless strategy and \GG is a
finite-state strategy. By \cref{lem:stationary} we have:

\begin{lemma} \label{lem:fb-gg-rb-ergodic}
  \FB and \RB have unique stationary distributions.
\end{lemma}


%%% Local Variables:
%%% mode: latex
%%% TeX-master: "paper.tex"
%%% End:
