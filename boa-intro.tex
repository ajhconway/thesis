\section{Introduction}\label{sec:boa-intro}

Dictionaries are among the most heavily used data structures. A
dictionary maintains a collection of key-value pairs
$\mathcal{S}\subseteq mathcal{U}\times mathcal{V}$, under operations\footnote{We do
  not consider dictionaries that also support the \Dsucc{x}{\mathcal{S}} and
  \Dpred{x}{\mathcal{S}}.  \Dsucc{x}{mathcal{S}} returns
  $\min\{y| y>x \land y\in\mathcal{S}\}$ and \Dpred{x}{mathcal{S}} is defined
  symmetrically.}  \Dinsert{x}{v}{\mathcal{S}}, \Ddelete{x}{mathcal{S}}, and
\Dquery{x}{\mathcal{S}}, which returns the value corresponding to $x$ when
$x\in S$.  When data fits in memory, there are many solutions to the
dictionary problem.

When data is too large to fit in memory, comparison-based dictionaries
can be quite varied.  They include the
\bet~\cite{DBLP:conf/soda/BrodalF03}, the write-optimized skip
list~\cite{DBLP:conf/pods/BenderFJMMPX17}, and the cache-optimized
look-ahead array
(COLA)~\cite{DBLP:conf/spaa/BenderFFFKN07,DBLP:journals/pvldb/BenderFJKKMMSSZ12,DBLP:conf/esa/BenderCDF02}.
These are optimal in the \defn{external-memory comparison model} in
that they match the bound established by Brodal and
Fagerberg~\cite{DBLP:conf/soda/BrodalF03} who showed that for any
dictionary in this model, if insertions can be performed in
$O\left(\frac{\grow\log_{\grow} N}{B}\right)$ amortized IOs, then
there exists a query that requires at least $\Omega(\log_\grow N)$
IOs, where $N$ is the number of items that can be stored in the data
structure, $B$ is the size of a memory transfer, and $\lambda$ is a
tuning parameter.  In the following $M$ will be the size of memory,
and $B = \Omega(\log n)$.  This trade off has since been extended in
several
ways~\cite{DBLP:conf/esa/BenderFGMMT14,DBLP:conf/soda/AfshaniBFFGT17}.

Iacono and P\v{a}tra\c{s}cu showed that in the DAM model, in which
operations beyond comparisons are allowed on keys, a better
tradeoff exists:
\begin{theorem}[\cite{DBLP:conf/soda/IaconoP12}]\label{thm:ip-lower-bound}
  If insertions into an external memory dictionary can be performed in
  $O\left(\grow/B\right)$ amortized IOs, then queries require an
  expected $\Omega(\log_\grow N)$ IOs.
\end{theorem}

They further describe an external-memory hashing algorithm, which we refer to
here as the \defn{IP hash table}, that performs insertions in
$O\left(\frac{1}{B}\left(\lambda + \log_{\frac{M}{B}} N + \log\log
N\right)\right)$ IOs and queries in $O(\log_\lambda N)$ IOs w.h.p. Therefore,
for $\lambda=\Omega\left(\log_{M/B}N + \log\log{N}\right)$, the IP hash table
meets the tradeoff curve of \Cref{thm:ip-lower-bound} and is thus optimal.

In dictionaries that do not support successors and predecessors, we
can assume that keys are hashed, that is, that they are uniformly
distributed and satisfy some independence properties. The IP hash
table and the following results hash all keys before
insertion and query in the dictionary by a
$\Theta(\log N)$-independent hash function.

The base result of this paper is a simple external-memory
hashing scheme, the \defn{Bundle of Arrays Hash Table} (BOA),
that meets the optimal \Cref{thm:ip-lower-bound} trade off curve for
large enough $\grow$. Specifically, we show:

\begin{restatable}{theorem}{boacost}\label{thm:boa-cost}
	A \boa{} supports $N$ insertions and deletions with amortized
	per entry cost of $O\left(\left(\grow + \log_{\frac{M}{B}} N +
	\log_\grow{N} \right)/B\right)$ IOs, for any $\grow > 1$. A query for a key
	$K$ costs $O(D_K\log_\grow{N})$ IOs w.h.p., where $D_K$ is the number of times
	$K$ has been inserted or deleted.
\end{restatable}

Thus BOAs are optimal for $\grow = \Omega(\log_{\frac{M}{B}} N +
\log_\grow{N})$.  They are readily modified to provide several variations,
notably the \defn{Bundle of Trees Hash Table} (BOT). BOTs are optimal for the
same range of $\grow$ as the IP hash table:

\begin{restatable}{theorem}{botcost}\label{thm:bot-cost}
	A BOT supports $N$ insertions and deletions with amortized per entry cost
	of $O\left(\left(\grow + \log_{\frac{M}{B}} N + \log\log M\right)/B\right)$
	IOs for any $\grow > 1$.  A query for a key $K$ costs $O(D_K\log_\grow{N})$
	IOs w.h.p., where $D_K$ is the number of times $K$ has been inserted or
	deleted.
\end{restatable}

We further introduce the first cache-oblivious hash table, the
\defn{Cache-Oblivious Bundle of Trees Hash Table} (COBOT), which matches the IO
performance of BOTs and IP hash tables.

\hanna{pink=needs fixing, yellow=fixed but check its ok, blue=fixed, green=rethink grammarwise}

The BOT can also be adapted to models in which disk reads and writes incur
different costs. The $\beta$-asymmetric BOT adjusts the merging schedule of a
regular BOT to trade some writes for more reads.

\begin{restatable}{theorem}{asymbot}\label{thm:bot-asymmetric}
	A $\beta$-asymmetric BOT supports $N$ insertions and deletions with
	amortized per entry cost of $O\left(\frac{1}{B}\left(\lambda +
	\frac{1}{\beta}\log_{\grow}{N} \right)\right)$ writes and
	$O\left(\frac{1}{B}\left(\lambda + \beta\right)\right)$ reads for any
	$\lambda > 1$ and $\beta \leq
	\left\lfloor\log_\lambda\frac{M}{B\log_\lambda N}\right\rfloor$. A query
	for a key $K$ performs $O(D_K\log_\lambda N)$ reads, where $D_K$ is the
	number of times $K$ has been inserted or deleted.
\end{restatable}
