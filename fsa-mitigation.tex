\def\checkmark{\tikzsetnextfilename{checkmark}\tikz\fill[scale=0.4](0,.35) -- (.25,0) -- (1,.7) -- (.25,.15) -- cycle;} 

\setlength\tabcolsep{3pt}
\begin{table}
\footnotesize
\centering
	\begin{tabular}{p{1in}*{6}{c}}
{\bf Feature} & {\bf \btrfs} & {\bf \ext} & {\bf \ftwofs} & {\bf \xfs} & {\bf \zfs} & {\bf \betrfs}\\
\hline
\raggedright Grouped allocation within directories & \checkmark & \checkmark & & \checkmark & \checkmark & \checkmark \\
Extents & \checkmark & \checkmark & & \checkmark & \checkmark &    \\
Delayed allocation & \checkmark & \checkmark & \checkmark & \checkmark & \checkmark & \checkmark \\
Packing small files & \checkmark & & & &  & \checkmark \\
 and metadata & (by OID) & & & & & \\
\hline
Default Node Size & 16 K & 4 K  & 4 K &  4 K &   8 K & 2--4 M\\
Maximum Node Size & 64 K & 64 K & 4 K & 64 K & 128 K & 2--4 M \\
Rewriting for locality & & & & & & \checkmark\\
Batching writes to reduce amplification & & & \checkmark & & & \checkmark \\
\end{tabular}
\caption{\label{tab:heuristics} \small
Principal anti-aging features of the file systems measured in this paper.
The top portion of the table are commonly-deployed features, and the bottom portion
indicates features our model (\S\ref{sec:seq}) indicates are essential;
an ideal node size should match the natural transfer size, which is roughly 4 MiB for 
modern HDDs and SSDs. OID in \btrfs is an object identifier, roughly corresponding to an inode number, which is assigned at creation time.}
\end{table}

\subsection{Existing Strategies to Mitigate Aging}\label{sec:fsa-filesystem}

When files are created or extended, blocks must be allocated to store the new
data.  Especially when data is rarely or never relocated, as in an
update-in-place file system like \ext{},
%Especially for update-in-place file systems, such as \ext{}, where data is
%rarely relocated,
initial block allocation decisions determine
% \mfc{``essential to'' has a positive note.  What I think we really mean is
% ``determine''}
performance over the life of the file system. Here we outline a few of the
strategies use in modern file systems to address aging, primarily at
allocation-time (also in the top of Table~\ref{tab:heuristics}).

\paragraph{Cylinder or Block Groups.}
FFS~\cite{DBLP:journals/tocs/McKusickJLF84} introduced the idea of \defn{cylinder groups},
which later evolved into block groups or allocation groups (\xfs).
Each
group maintains information about its inodes and a bitmap of blocks. A
new directory is placed in the cylinder group that contains more than the
average number of free inodes, while inodes and data blocks of files in one
directory are placed in the same cylinder group when possible.

\zfs~\cite{DBLP:conf/lisa/Bonwick07a} is designed to pool
storage across multiple devies~\cite{DBLP:conf/lisa/Bonwick07a}.
%%% When allocating space for a file, there are three decisions:
%%% device selection, metaslab selection, and block selection
%%% Device selection tries to spread the data across
%%% all devices in the pool so that maximum bandwidth can be obtained;
%%% note that this is not just space, but also factors that 
%%% impact throughput, like bit density and angular velocity.
\zfs selects from one of a few hundred \defn{metaslabs} on a 
device, based on a weighted calculation of several factors
including minimizing seek distances.
%such as reusing the same metaslabs to minimize seek distance.
The metaslab with the highest weight is chosen.
%%% Within a metaslab, space is managed by a \defn{space map},
%%% which tracks available extents using an AVL tree.
%%% zfs chooses a block within the meta-slab with a
%%% variant of first-fit policy.  

In the case of \ftwofs~\cite{DBLP:conf/fast/LeeSHC15}, a log-structured file system, 
the disk is divided into segments---the granularity at which the log is garbage collected, or cleaned.
The primary locality-related optimization in \ftwofs is that writes
are grouped to improve locality, and dirty segments are filled before
finding another segment to write to. In other words, writes with temporal
locality are more likely to be placed with physical locality.

Groups are a best-effort approach to directory locality:
space is reserved for co-locating files in the same directory,
but when space is exhausted, files in the same directory can be scattered
across the disk.  Similarly, if a file is renamed, it is not physically moved
to a new group.

\paragraph{Extents.} All of the file systems we measure, except \ftwofs and
\betrfs,
%\mfc{the table says we do}, 
allocate space using \defn{extents}, or runs of physically contiguous blocks.
In \ext~\cite{ext2,ext3,DBLP:journals/usenix-login/MathurCD07}, for example, an
extent can be up to 128 MiB.  Extents reduce bookkeeping overheads (storing a
range versus an exhaustive list of blocks).  Heuristics to select larger
extents can improve locality of large files.  For instance, \zfs selects from
available extents in a metaslab using a first-fit policy.
%tracks available extents in a metaslab using an AVL tree, and selects an
%extent using first-fit policy.

\paragraph{Delayed Allocation.} Most modern file systems, including \ext, \xfs,
\btrfs, and \zfs, implement delayed allocation, where 
logical blocks are not allocated until buffers are written to disk. 
By delaying allocation when a file is growing,
the file system can allocate a larger extent for data appended to the same file.
%This offers the opportunity to allocate larger extents 
%unneccessary temporary file allocations, and it aggregates related allocation requests.
However, allocations can only be delayed so long without
violating durability and/or consistency requirements; a typical
file system ensures data is dirty no longer than a few seconds.
Thus, delaying an allocation only improves locality inasmuch
as adjacent data is also written on the same timescale;
delayed allocation alone cannot prevent fragmentation when 
data is added or removed  over larger timescales.
%in order to maintain file system consistency and to
%prevent data loss, most file systems ensure that all data (including unallocated
%data) is written to disk every few seconds; it is unclear how
%the extent to which
%this heuristic improves allocations for operations that span
%larger timescales.

Application developers may also request a persistent preallocation
of contiguous blocks using fallocate.
%\mfc{I thought we weren't
%  using this font.  And we don't just two sentences on.}
%which is a useful interface to tell the file system to
%allocate contiguous blocks for a given file at its best.
To take full advantage of this interface, 
developers must know each file's size in advance. % in order to take advantage of it.
Furthermore, fallocate can only help intrafile fragmentation; there is currently 
not an
analogous interface to ensure directory locality.

\paragraph{Packing small files and metadata.}
For directories with many small files, an important optimization 
can be to pack the file contents, and potentially metadata,
into a small number of blocks or extents.
\btrfs~\cite{DBLP:journals/tos/RodehBM13} stores metadata of files and
directories in copy-on-write B-trees. Small files are broken into one or more
fragments, which are packed inside the B-trees.  For small files, the fragments
are indexed by object identifier (comparable to inode number); the locality of
a directory with multiple small files depends upon the proximity of the object
identifiers.

\betrfs stores metadata and data as key-value
pairs in two \bets.  Nodes in a \bet are large (2--4 MiB),
amortizing seek costs.  Key/value pairs are packed
within a node by sort-order, and nodes are periodically rewritten,
copy-on-write, as changes are applied in batches.

\betrfs also divides the namespace of the file system into \defn{zones} of a
desired size (512 KiB by default), in order to maintain locality within a
directory as well as implement efficient renames.  Each zone root is either a
single, large file, or a subdirectory of small files.
%; each zone is a file or subdirectory.
The key for a file or directory is its relative path to its zone root. The
key/value pairs in a zone are contiguous, thereby maintaining locality.
