\section{Cache-Oblivious BOTs}\label{sec:boa-cacheoblivious}

In this section, we show how to modify a BOT to be cache oblivious.  We call
the resulting structure a cache-oblivious hash tree (COBOT). 

Much of the structure of the BOT translates directly into the cache-oblivious
model. However, some changes are necessary. In particular, when the series of
character queues are merged, this merge must be performed cache-obliviously
using funnels~\cite{DBLP:conf/focs/FrigoLPR99}, rather than with an (up to)
$M/B$-way merge. Also, the log cannot be buffered into sections of size $O(B)$,
and so instead they are buffered into sections of constant size, items are
immediately added to the routing filter, and the extra IOs are eliminated by
optimal caching.

When an insertion is made into a COBOT, its fingerprint-value pair is
appended to the log, and it is immediately inserted into level 1. Thus, the
leaves of the routing trees point to single entries in the log.

The series of the character queues must be placed more carefully as well. In
particular the runs of series $\sigma_j$ must be laid out back-to-back for all
$j$ (rather than just small $j$ as in \Cref{sec:character-queue}), so that the
caching algorithm can buffer them appropriately.

The series are merged using a \defn{partial funnelsort}. Funnelsort is a
cache-oblivious sorting algorithm that makes use of
$K$-funnels~\cite{DBLP:conf/focs/FrigoLPR99}. A $K$-funnel is a CO data structure
that merges $K$ sorted lists of total length $N$. We make use of the the
following lemma.

\begin{lemma}[\cite{DBLP:conf/focs/FrigoLPR99}]\label[lemma]{lem:funnel}
	A $K$-funnel merges $K$ sorted lists of total length $N\geq K^3$ in
	$O\left(\frac{N}{B}\log_{M/B} \frac{N}{B} + K + \frac{N}{B}\log_{K}
	\frac{N}{B}\right)$ IOs, provided the tall cache assumption that
	$M=\Omega(B^2)$ holds.
\end{lemma}
	
The partial funnelsort used to merge $K$ runs of a series with total length $L$
(in words) performs a single merge with a $K$-funnel if $L \geq K^3$, and
recursively merges the run in groups of $K^{1/3}$ runs otherwise.

\begin{corollary}\label[corollary]{cor:funnelsort}
	A partial funnelsort merges $K$ runs of total word length $L$ in
	$O\left(\frac{L}{B}\log_{M/B} \frac{L}{B} + \frac{L}{B}\log_{K}
	\frac{L}{B}\right)$ IOs, provided the tall cache assumption that
	$M=\Omega(B^2)$ holds.
\end{corollary}

\begin{proof}
	The base case of the recursion occurs either when there is only 1 list
	remaining or the remaining lists fit in memory. In any other case of the
	recursion, since $L=\Omega(B^2)$ by the tall cache assumption, the $K$ term
	in \Cref{lem:funnel} is dominated.

	The recurrence is dominated by the cost of the funnel merges, which yields
	the result.
\end{proof}

\begin{theorem}
	If $M=\Omega(B^2)$, then a COBOT with $N$ entries and growth factor
	$\lambda$ has amortized insertion/deletion cost
	$\Theta\left(\frac{1}{B}\left(\lambda + \log\log M + \log_{M/B}
	N/B\right)\right)$. A query for key $K$ has cost
	$\Theta\left(D_K\log_\lambda N\right)$, w.h.p., where $D_K$ is the
	duplication count of $K$.
\end{theorem}

\begin{proof}
	We may assume that the caching algorithm sets aside enough memory that the
	last $B$ items in the log, together with the subtree rooted at their least
	common ancestor, are cached. Thus the log is updated at a per-item cost of
	$O(1/B)$.

        The proof of \Cref{thm:bot-cost} now carries over to the COBOT. The
        routing filters are updated the same way, and the cost of updating the
        character queues is unchanged, by \Cref{cor:funnelsort}.

	Queries are performed as in \Cref{sec:routing-tree-query}, except that now
	the level 1 nodes cover $O(1)$ fingerprints, but the depth of the tree is
	unchanged, so the cost is the same.
\end{proof}
