\section{Bundle of Trees Hash Tables in the Asymmetric Model}\label{sec:boa-asymmetric}

In this section we describe the \defn{$\beta$-asymmetric Bundle of Arrays Hash
table} (Asymmetric BOT), which adapts BOTs to the asymmetric external memory
model~\cite{DBLP:conf/spaa/BlellochFGGS15}. This model is similar to the
regular external memory model of Aggarwal and Vitter, except that the cost of
reading a block is 1, but where the cost of writing a block is $\omega>1$.

The underlying idea is to not write character on most levels and instead read
the necessary information from the queues of descendant nodes. This only
requires reads, provided no more than $M/B$ such queues are read at a time.
Thus, every $\beta \leq \left\lfloor\log_\lambda\frac{M}{B\log_\lambda
N}\right\rfloor$ levels, all the character queues must be merged and stored. We
say level $i$ is a \defn{queue level} if $i$ is divisible by $\beta$.

When the $i$th level root fills, where level $i$ is not a queue level, it is
added to level $i+1$. For each key $K$ covered by the now-full orphan root, the
high-order bits $\high{i+1}{K}$, check characters $C_{i+1}(K)$ and next
characters $D_{i+2}(K)$ are obtained by merging the character queues on the
highest queue level below $i$, which is $\ell=i - \left(i\bmod\beta\right)$.
Thus $L_\ell^{\ell + 1},\ldots, L_\ell^{i + 1}$ are read;  for each key $K$ in
sorted order this yields: $\high{\ell}{K}$ together with the characters
$D_{\ell + 1}(K),\ldots,D_{i+2}(K)$, from which $\high{i+1}{K}$ and
$D_{i+2}(K)$ can be computed, as well as the check character $C_{i+1}(K)$. This
computation occurs sequentially, so that the intermediate results need not be
written out.

When the $i$th level root fills on a queue level, its character queues are
created before it is added to level $i+1$. This involves computing
$\high{i}{K}$ in sorted order as above and then merging all the character
queues $L_{i-\beta}^{i+1},L_{i-\beta}^{i+2},\ldots$ into new character queues
$L_i^{i+1},L_i^{i+2},\ldots$. These merges are performed with a single
$\lambda^\beta$-way merge. Then the now-full level root can be added to the
routing filter of the $(i+1)$st level root using the character queue
$L_i^{i+1}$.

We refer to this modified BOT as an \defn{$\beta$-asymmetric BOT}. 

\asymbot*

\begin{proof}
	Each insertion will eventually be added to each level $i$. When the level
	$i$ is not a queue level, $i\bmod \beta$ characters per item will be read from
	the character queues on the queue level below $i$. Summed over all levels, this
	yields
	\[\sum_{i=0}^{\log_\lambda N} \frac{i\bmod \beta}{B\log_\lambda N} = O\left(\frac{\beta}{B}\right)\]
	reads per insertion.

	A $\lambda^\beta$-way merge is performed every $\beta$ levels. Since $\beta
	\leq \left\lfloor\log_\lambda\frac{M}{B\log_\lambda N}\right\rfloor$, this
	requires $O(1/B)$ reads and writes per element. This contributes
	$O\left(\frac{\log_\grow{N}}{\beta B}\right)$ reads and writes per
	insertion in total.

	The per-insertion read and write cost of building the routing filters is
	$\Theta(\lambda/B)$ as in the proof of \Cref{thm:bot-cost}.

	The cost per query is the same as in \Cref{thm:bot-cost}.
\end{proof}

\Cref{thm:bot-asymmetric} can improve the write cost when $\grow =
o(\log\log M)$ and $\log\log M = \Omega\left(\log_{\frac{M}{B}}{N}\right)$. In
that case, $\beta$ can be tuned to optimize insertion performance relative to
the $\omega$ of the AEMM by solving the quadratic equation $\beta = \omega
\cdot \left(\lambda + \frac{1}{\beta}\log_{\grow}{N} \right)$. 

In particular, an interesting consequence of \Cref{thm:bot-asymmetric} is
in the case where $N$ is polynomial in $M$. Then insertions can be performed
into a $(\log_\grow{M/B})$-asymmetric BOT with growth factor $\lambda = O(1)$
with constant write amplification. This does come at the cost of more reads
than when using a regular BOT:

\begin{corollary}
        If $N = O(M^c)$ for some constant $c$, then a
        $(\log_\grow{M/B})$-asymmetric BOT with growth factor $\lambda=O(1)$
        containing $N$ key-value values performs $\Theta(1/B)$ amortized writes
        per insertion, $\Theta\left(\frac{1}{B}\left(\lambda +
        \log{\frac{M}{B}}\right)\right)$ amortized reads per insertion and
        $\Theta(\log N)$ reads per query.
\end{corollary}
