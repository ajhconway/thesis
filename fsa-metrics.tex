\section{Measuring File System Fragmentation}\label{sec:fsa-metrics}

This section explains the two measures for file system fragmentation used in
our evaluation: recursive scan latency and dynamic layout score, a modified
form of Smith and Seltzer's layout score~\cite{DBLP:conf/sigmetrics/SmithS97}.
These measures are designed to capture both intra-file fragmentation and
inter-file fragmentation. 

\paragraph{Recursive grep test.} One measure we present in the following
sections is the wall-clock time required to perform a recursive grep in the
root directory of the file system.  This captures the effects of both inter-
and intra-file locality, as it searches both large files and large directories
containing many small files.  We report search time per unit of data,
normalizing by using \ext's du output.  We will refer to this as the grep test. 

\paragraph{Dynamic layout score.}  Smith and Seltzer's layout
score~\cite{DBLP:conf/sigmetrics/SmithS97} measures the fraction of blocks in a
file or (in aggregate) a file system that are allocated in a contiguous
sequence in the logical block space.  We extend this score to the dynamic I/O
patterns of a file system.  During a given workload, we capture the logical
block requests made by the file system, using blktrace, and measure the
fraction that are contiguous.  This approach captures the impact of placement
decisions on a file system's access patterns, including the impact of metadata
accesses or accesses that span files.  A high dynamic layout score indicates
good data and metadata locality, and an efficient on-disk organization for a
given workload.

One potential shortcoming of this measure is that it does not distinguish
between small and large discontiguities. Small discontiguities on a hard drive
should induce fewer expensive mechanical seeks than large discontiguities in
general, however factors such as track length, difference in angular placement
and other geometric considerations can complicate this relationship. A more
sophisticated measure of layout might be more predictive.  We leave this for
further research. On SSD, we have found that the length of discontiguities has
a smaller effect.  Thus we will show that dynamic layout score strongly
correlates with grep test performance on SSD and moderately correlates on hard
drive.
