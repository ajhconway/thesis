\section{Evaluation}\label{sec:eval}

We evaluate the performance of \sysname{} on several microbenchmarks and on the
standard YCSB application benchmark\cite{DBLP:conf/cloud/CooperSTRS10}. We
compare this performance against that of two state-of-the-art key-value stores,
RocksDB and PebblesDB.  The following questions drive our evaluation:
%In our evaluation, we first wish to demonstrate \sysname's macro performance.
%Then, we would like to understand the performance impacts of several specific
%design decisions in \sysname, and to quantify both the advantages and drawbacks
%of these choices. Specifically we hope to answer the following questions:
\begin{itemize}
   \item Does \sysname achieve its primary goal of improved insertion
      performance?
   \item To what extent is this performance achieved through reduced write
      amplification as opposed to other factors?
   \item Does increasing insert performance come at a cost to [range] query
     performance?  In particular, do queries in \sysname require more I/O, due to
     size tiering, than in non-size-tiered systems?
   \item Are sequential (or otherwise local) insertions faster as predicted on
     \sysname? Do they have lower write amplification?
   \item Can \sysname utilize device bandwidth for large range queries?
   \item \sysname is designed to be highly concurrent; do point lookups scale
      with the number of threads?
\end{itemize}

\subsection{Setup and Workloads}\label{sec:setup}

All results are collected on a Dell PowerEdge R720 with a 32-core
2.00 GHz Intel Xeon CPU, 256 GiB RAM and a 960GiB Intel Optane 905p PCI Express
3.0 NVMe device. The block size used was 4096 bytes.

In general, we use workloads derived from YCSB traces with 24B keys. We
generally use 100B values, but also include a set of YCSB benchmarks for 1KiB
values.  We instrumented dry runs of YCSB in order to collect workload traces
for the load and $A-F$ YCSB workloads and replay them on each of the databases
evaluated.  In order to eliminate the overhead of reading from a trace file
during the experiment, the trace replayer \texttt{mmap}s the trace file before
starting the experiment.  We use the same traces for each system.

In general, we limit the available memory to 10\% of the dataset size or less.
In order to perform the benchmarks on reasonably sized datasets, we restrict
the available system memory with a type 1 Linux \texttt{cgroup}, sized to the
target memory size plus the size of the trace, which we pin so that it cannot
be swapped out. Unless otherwise noted, the target memory size is 4GiB.
PebblesDB has an apparent memory leak, which causes it to consume the available
memory, so we allow it to use the full system memory. On the YCSB load
benchmarks, this causes it to swap for a small portion at the end, but this was
less than 10\% of the  run time.

Unless otherwise noted, \sysname{} uses a max fanout of 8, a memtable size of
24MiB and a total cache size of 3.25GiB. The difference between this cache size
and the target memory size of 4GiB is to accommodate other in-memory data
structures maintained by \sysname.

Each system is run with the thread count which yields the highest throughput.
RocksDB is configured to use background threads equal to the number of cores
minus the number of foreground threads, with a minimum of 4. PebblesDB uses its
default number of background compaction threads. \sysname is configured without
background compaction threads.

\subsection{YCSB}\label{sec:ycsb}

We measure application performance using the Yahoo Cloud Services Benchmark
(YCSB). The core YCSB workloads consist of load phases and run phases.
The load phases create a dataset by inserting uniformly random
key-value pairs. 
The run phases emulate various workload mixes. Workload A is 
50\% updates, 50\% reads, workload B is 95\% reads, 5\%
updates), workload C is 100\% reads, workload D is read latest (95\%
reads, 5\% insertions), workload E is short range scans (95\% scans, 5\% insertions)
and workload F is read-modify-writes (50\% reads, 50\% RMWs).

\begin{figure*}
      \centering
      \ref{ycsb-legend} \\
      \begin{subfigure}{0.54\linewidth}
         \begin{tikzpicture}
            \begin{axis}[
                  ybar=1pt,
                  bar width=5.5pt,
                  enlarge x limits=0.1,
                  width = \hsize,
                  height = 2.5in,
                  symbolic x coords = {Load, A, B, C, D, E, F},
                  xtick = data,
                  major x tick style = transparent,
                  ymin = 0,
                  ymax = 2750,
                  yticklabel style = {rotate=90},
                  ylabel near ticks,
                  /pgf/number format/1000 sep={},
                  nodes near coords,
                  every node near coord/.append style={rotate=90, anchor=west, font=\tiny},
                  ylabel={Operations/Second (Thousands)},
                  xlabel={YCSB Workload (24B keys, 100B values)},
                  label style={font=\small, align=center},
                  legend columns=5,
                  %legend style={at={(0.98, 0.98)}, anchor=north east}
                  legend to name={ycsb-legend}
               ]
               \addplot[style={Plum,fill=Plum,mark=none} ]
               table [col sep=space] {data/ycsb-100b/splinterdb.csv};
               \addplot[style={RoyalBlue,fill=RoyalBlue,mark=none}]
               table [col sep=space] {data/ycsb-100b/rocksdb.csv};
               \addplot[style={ForestGreen,fill=ForestGreen,mark=none}]
               table [col sep=space] {data/ycsb-100b/pebblesdb.csv};
               \legend{\sysname, RocksDB, PebblesDB};
            \end{axis}
         \end{tikzpicture}
         \caption{Throughput on YCSB workloads with 24B keys and 100B values.
         Load is 673M operations, E is 20M operations and others are 160M
         operations. Higher is better.}\label{fig:ycsb-100B}
      \end{subfigure}
      \hfill
      \begin{subfigure}{0.44\linewidth}
         \centering
            \begin{tikzpicture}
               \begin{axis}[
                     ybar=1pt,
                     bar width=5.5pt,
                     enlarge x limits=0.1,
                     width = \hsize,
                     height = 2.5in,
                     symbolic x coords = {Load, A, B, C, D, E, F},
                     xtick = data,
                     major x tick style = transparent,
                     ymin = 0,
                     ymax = 900,
                     yticklabel style = {rotate=90},
                     ylabel near ticks,
                     /pgf/number format/1000 sep={},
                     nodes near coords,
                     every node near coord/.append style={rotate=90, anchor=west, font=\tiny},
                     ylabel={Operations/Second (Thousands)},
                     xlabel={YCSB Workload (24B keys, 1KiB values)},
                     label style={font=\small, align=center},
                     legend columns=1,
                     legend style={at={(0.98, 0.98)}, anchor=north east}
                     %legend to name={ycsb-legend}
                  ]
                  \addplot[style={Plum,fill=Plum,mark=none} ]
                  table [col sep=space] {data/ycsb-1kib/splinterdb.csv};
                  \addplot[style={RoyalBlue,fill=RoyalBlue,mark=none}]
                  table [col sep=space] {data/ycsb-1kib/rocksdb.csv};
                  %\addplot[style={ForestGreen,fill=ForestGreen,mark=none}]
                  %table [col sep=space] {data/ycsb-1kib/pebblesdb.csv};
                  %\legend{\sysname, RocksDB};%, PebblesDB};
            \end{axis}
         \end{tikzpicture}
         \caption{Throughput on YCSB workload with 24B keys and 1KiB values.
         Load is 84M operations, E is 1.3M operations and others are 10M
         operations. Higher is better.}\label{fig:ycsb-1KiB}
      \end{subfigure}
      \caption{YCSB throughput and I/O benchmark results.}
\end{figure*}

\begin{figure}
   \centering
   \begin{tikzpicture}
      \begin{axis}[
            ybar=3pt,
            bar width=24pt,
            enlarge x limits=0.25,
            width = \hsize,
            height = 2in,
            symbolic x coords = {load-write, load-total, c-read},
            xticklabel style = {align=center},
            xticklabels = {Load:\\Write Amp, Load:\\Total I/O Amp, Run C:\\Read Amp},
            xtick = data,
            %xticklabel style = {font=\small},
            major x tick style = transparent,
            ymin = 0,
            ymax = 37,
            ylabel={I/O Amplification},
            ylabel near ticks,
            xlabel={YCSB IO Amplification (24B keys, 100B values)},
            label style = {font=\small},
            nodes near coords,
            %every node near coord/.append style={font=\scriptsize},
            nodes near coords style={/pgf/number format/.cd, fixed zerofill, precision=1},
            %legend columns=1,
            %legend style={font=\scriptsize, at={(0.02, 0.98)}, anchor=north west},
         ]
         \addplot[style={Plum,fill=Plum,mark=none}]
         table [col sep=space] {data/ycsb-amp-100b/splinterdb.csv};
         \addplot[style={RoyalBlue,fill=RoyalBlue,mark=none}]
         table [col sep=space] {data/ycsb-amp-100b/rocksdb.csv};
         \addplot[style={ForestGreen,fill=ForestGreen,mark=none}]
         table [col sep=space] {data/ycsb-amp-100b/pebblesdb.csv};
         %\legend{\sysname, \sysname (12 threads), RocksDB, PebblesDB};
      \end{axis}
   \end{tikzpicture}
   \caption{IO amplification on YCSB load and Run C workloads, as
   measured with iostat. Lower is better.}\label{fig:ycsb-amp-100B}
\end{figure}

We perform the benchmark with 24B keys and two different value sizes
configurations: one with small 100B values, shown in
\cref{fig:ycsb-100B,fig:ycsb-amp-100B} and one with large 1KiB values, shown in
\cref{fig:ycsb-1KiB}.

On the load phase, \sysname is faster by almost an order of magnitude.  Because
of size-tiering and its compaction/flushing policy \sysname has about $1/2$ the
write amplification of the other systems. Note PebblesDB performs almost no
reads because it was given unlimited memory.  Surprisingly PebblesDB does not
show substantially lower write amplification than RocksDB.

On the run phases, which the exception of E, \sysname is 40--150\% faster than
RocksDB, the next fastest system. On E, \sysname is roughly half as fast as
RocksDB.


\subsection{KVell}\label{sec:kvell}

KVell~\cite{DBLP:conf/sosp/LepersBGZ19} is a key-value store also designed to
utilize full NVMe bandwidth.  It has an in-memory B-tree index that maps all
keys to disk page offsets.  It does well on large (4KiB) key-value pairs, but
on small key-value pairs, the overhead of the in-memory index becomes a
significant fraction of the dataset size.  In particular, it was impossible to
run KVell in a memory \texttt{cgroup} of 4GiB.  Figure~\ref{fig:kvell} shows
KVell's performance on the YCSB workload with 100B values, for different memory
sizes. At 22GiB, which is around the size of the in-memory index, KVell's
performance starts to drop. At 20GiB, KVell becomes unusable. Therefore in
realistic memory settings, KVell is not a viable option for the small key-value
sizes that \sysname targets.

\begin{figure}
   \begin{tikzpicture}
      \begin{axis}[
            ybar=1pt,
            bar width=8pt,
            enlarge x limits=0.1,
            width = \hsize,
            height = 3in,
            symbolic x coords = {Load, A, B, C, D, E, F},
            xtick = data,
            major x tick style = transparent,
            ymin = 0,
            ymax = 3700,
            yticklabel style = {rotate=90},
            ylabel near ticks,
            /pgf/number format/1000 sep={},
            nodes near coords,
            every node near coord/.append style={/pgf/number format/fixed, rotate=90, anchor=west, font=\small},
            ylabel={Operations/Second (Thousands)},
            xlabel={YCSB Workload (24B keys, 100B values)},
            %label style = {font=\small, align=center},
            legend columns=5,
            legend style={at={(0.98, 0.98)}, anchor=north east}
            %legend to name={ycsb-legend}
         ]
         \addplot[style={Red,fill=Red!25!white,mark=none} ]
         table [col sep=space] {data/ycsb-100b/kvell20.csv};
         \addplot[style={Red,fill=Red!50!white,mark=none} ]
         table [col sep=space] {data/ycsb-100b/kvell22.csv};
         \addplot[style={Red,fill=Red,mark=none} ]
         table [col sep=space] {data/ycsb-100b/kvell24.csv};
         \addplot[style={Red,fill=Red!50!black,mark=none} ]
         table [col sep=space] {data/ycsb-100b/kvell26.csv};
         \addplot[style={Red,fill=Red!25!black,mark=none} ]
         table [col sep=space] {data/ycsb-100b/kvell28.csv};
         \legend{20GiB, 22GiB, 24GiB, 26GiB, 28GiB};
      \end{axis}
   \end{tikzpicture}
   \caption{Throughput of Kvell on YCSB workloads with varying amounts of
   available RAM. Load consists of 673M operations, E consists of 20M
   operations and all other workloads consist of 160M operations. Higher is
   better.}\label{fig:kvell}
\end{figure}

\subsection{Sequential Insertion Performance}\label{sec:seq}

Because \sysname{} is based on an \datastruct{} and makes use of a
flush-then-compact policy, we predict that its performance will
improve substantially on insertion workloads with a high degree of
locality (see \cref{sec:flush-compact}). We test this hypothesis by
performing 20GiB of single-threaded insertions from a trace composed of
interleaved sequential and random keys in different proportions. For
comparison, we perform the same workload on RocksDB.

As shown in \cref{fig:sequential}, \sysname's performance improves smoothly
from 349K insertions per second for a purely random workload to 614K insertions
per second for a purely sequential workload, which is 76\% faster. This
improvement is partially obscured by the log, which adds a constant additive IO
overhead. If we disable the log, \sysname improves from 430K insertions per
second on a purely random workload to 866K operations per second on a purely
sequential workload, 100\% faster. Note that we would expect the intermediate
throughputs in the best case to be the [weighted] harmonic mean of the pure
cases, because they are rates. At 50\% random, 50\% sequential for \sysname
with no log this is 575K insertions/second, so its actual performance of 521K
insertions/second captures a substantial amount of the potential improvement.

RocksDB also improves as the workload becomes more sequential, but this effect
is much smaller, a 35\% speedup. Furthermore, RocksDB shows less than 20\%
speedup until the workloads becomes 99\% sequential.

\Cref{fig:sequential-amp} shows that as predicted, \sysname incurs less IO
amplification on more sequential workloads. With the log disabled, its write
amp approaches 1 as the workload approaches purely sequential.  In contrast,
while RocksDB also has less IO amplification on more sequential workloads, it
still incurs write amplification of 4.1 even when 99\% of the keys are
sequential. It is only when the workload becomes 100\% sequential that the
write amplification becomes close to 1 (because of caching it even falls below
1).

\begin{figure}
   \begin{tikzpicture}
      \begin{axis} [
            width = \hsize,
            height = 2.75in,
            xtick = data,
            ytick distance = 500,
            minor y tick num = 4,
            enlarge x limits = 0.05,
            symbolic x coords = {0, 50, 90, 99, 100},
            %major x tick style = transparent,
            ymin = 0,
            ymax = 1000,
            ylabel={Operations/Second (Thousands)},
            yticklabel style = {rotate=90},
            ylabel near ticks,
            %label style = {font=\small, align=center},
            xmin = 0,
            xmax = 100,
            nodes near coords,
            xlabel={Percentage Sequential Insertions},
            %legend cell align={left},
            %legend to name=seq-legend,
            legend columns=1,
            legend style={at={(0.02, 0.98)}, anchor=north west}
         ]
      \addplot [
         style={Plum},
         line width=1.5pt,
         mark=*,
         visualization depends on=\thisrow{Alignment} \as \alignment,
         %visualization depends on=\thisrow{YShift} \as \yshift,
         every node near coord/.append style={anchor=north}
      ]
         table [
            col sep=space,
         ] {data/sequential/splinterdb.csv};
      \addplot [
         style={VioletRed},
         line width=1.5pt,
         mark=*,
         %visualization depends on=\thisrow{Alignment} \as \alignment,
         %visualization depends on=\thisrow{YShift} \as \yshift,
         every node near coord/.append style={anchor = south},
      ]
         table [
            col sep=space,
         ] {data/sequential/splinterdb-no-log.csv};
      \addplot [
         style={RoyalBlue},
         line width=1.5pt,
         mark=square*,
         visualization depends on=\thisrow{Alignment} \as \alignment,
         %visualization depends on=\thisrow{YShift} \as \yshift,
         every node near coord/.append style={anchor=south}
      ]
         table [
            col sep=space,
         ] {data/sequential/rocksdb.csv};
         \legend{\sysname, \sysname (no log), RocksDB};
      \end{axis}
   \end{tikzpicture}
   \caption{Single-threaded insertion throughput by locality. X-axis indicates
   the percentage of sequential keys.  X-axis not to scale. Higher is
   better.}\label{fig:sequential}
\end{figure}

\begin{figure}
   \begin{tikzpicture}
      \begin{axis} [
            width = \hsize,
            height = 2.75in,
            xtick = data,
            minor y tick num = 4,
            symbolic x coords = {0, 50, 90, 99, 100},
            enlarge x limits = 0.05,
            %major x tick style = transparent,
            ymin = 0,
            ymax = 15,
            ylabel={I/O Amplification},
            yticklabel style = {rotate=90},
            ylabel near ticks,
            %label style = {font=\small},
            xmin = 0,
            xmax = 100,
            xlabel={Percentage Sequential Insertions},
            %legend cell align={left},
            legend style={at={(0.98, 0.98)}, anchor=north east},
            legend columns = 1
         ]
      \addplot [
         style={Plum},
         line width=1pt,
         mark=*,
         nodes near coords,
         %visualization depends on=\thisrow{WriteAlignment} \as \alignment,
         %visualization depends on=\thisrow{WriteOpacity} \as \opacity,
         every node near coord/.append style={anchor=south, rotate = 0},
      ]
         table [
            col sep=space,
            y=Write,
         ] {data/sequential-amp/splinterdb.csv};
      \addplot [
         style={Plum},
         line width=1pt,
         mark=o,
         mark options={solid},
         %dashed,
         %nodes near coords,
         %visualization depends on=\thisrow{TotalAlignment} \as \alignment,
         %every node near coord/.append style={font=\scriptsize, anchor=\alignment},
         forget plot
      ]
         table [
            col sep=space,
            y=Total
         ] {data/sequential-amp/splinterdb.csv};
      \addplot [
         style={VioletRed},
         line width=1pt,
         mark=*,
         nodes near coords,
         %visualization depends on=\thisrow{WriteAlignment} \as \alignment,
         %visualization depends on=\thisrow{WriteOpacity} \as \opacity,
         every node near coord/.append style={anchor=north, rotate = 0},
      ]
         table [
            col sep=space,
            y=Write,
         ] {data/sequential-amp/splinterdb-no-log.csv};
      \addplot [
         style={VioletRed},
         line width=1pt,
         mark=o,
         mark options={solid},
         dashed,
         %nodes near coords,
         %visualization depends on=\thisrow{TotalAlignment} \as \alignment,
         %every node near coord/.append style={font=\scriptsize, anchor=\alignment},
         forget plot
      ]
         table [
            col sep=space,
            y=Total
         ] {data/sequential-amp/splinterdb-no-log.csv};
      \addplot [
         style={RoyalBlue},
         line width=1pt,
         mark=square*,
         nodes near coords,
         %visualization depends on=\thisrow{WriteAlignment} \as \alignment,
         %visualization depends on=\thisrow{WriteOpacity} \as \opacity,
         every node near coord/.append style={anchor=south, rotate = 0},
      ]
         table [
            col sep=space,
            y=Write
         ] {data/sequential-amp/rocksdb.csv};
      \addplot [
         style={RoyalBlue},
         line width=1pt,
         mark=square,
         mark options={solid},
         dashed,
         %nodes near coords,
         %visualization depends on=\thisrow{TotalAlignment} \as \alignment,
         %every node near coord/.append style={font=\scriptsize, anchor=\alignment},
         forget plot
      ]
         table [
            col sep=space,
            y=Total
         ] {data/sequential-amp/rocksdb.csv};
         \legend{\sysname, \sysname (no log), RocksDB};
      \end{axis}
   \end{tikzpicture}
   \caption{I/O amplification of mixed sequential/random insertion
     workloads. Shown are write amplification (solid) and
     total IO amplification (dashed) as measured with
     \texttt{iostat}. X-axis not to scale. Lower is
     better.}\label{fig:sequential-amp}
\end{figure}


\subsection{Concurrency Scaling}\label{sec:scaling}

\Sysname is designed to scale with the number of available cores up to the
performance limits of the storage device. This is especially true for reads,
where the use of distributed reader-writer locks and a highly concurrent cache
design, together with a careful avoidance of dirtying cache lines, can avoid
almost all contention between threads.

\paragraph{Read Concurrency}
We test the read concurrency scaling of \sysname{} by running YCSB
workload C with 160M key-value pairs, where (as in \cref{fig:ycsb-100B} each
instance of the test divides the keys into $N$ evenly divided batches, which
are then performed in parallel by $N$ threads. The results are in
\cref{fig:read-concurrency}.

The results show nearly linear scaling---throughput with 24 threads is
18.5$\times$ the single-threaded throughput. Between roughly 24 and 32 threads,
the scaling flattens out, but at that point the measured throughput is
2.07--2.24 GiB/sec, which is 88--95\% of the device's advertised random read
capability.

While RocksDB also scales well, its throughput with 24 threads is 17.4$\times$
its single-threaded throughput, and with 32 threads it uses 91\% of the
device's advertized random read capability. Therefore, even though \sysname can
perform more operations per second, RocksDB is still making nearly full use of
the device for reads. We conclude here that SplinterDB is making better use of
the available memory for caching, since it has noticeably lower read
amplification.

\alex{Say something about Pebbles?}

\paragraph{Insertion Concurrency}
We test the insertion concurrency scaling of \sysname by running the YCSB load
workload with 673M key-value pairs divided into $N$ batches, each of which
is inserted in parallel by a different thread. \cref{fig:write-concurrency}
shows selected $N$ for each system, including its peak throughput.

The results show that \sysname scales almost linearly up to 10 threads. With
10+ threads, it performs 2.0-2.4M insertions per second with IO amplification
around 7.5, which implies that it uses 1.9-2.2GiB/sec of bandwidth, which is at
or near the device's sequential bandwidth of 2.2GiB/sec.

RocksDB's insertion performance also scales as the number of threads increase
up to 14 threads, by a factor of 2.7. At its peak, it uses 754GiB/sec of
bandwidth. PebblesDB scales slightly as well.  For both RocksDB and PebblesDB,
as many background threads as available are used for flushing and compaction
during this benchmark.

\begin{figure}
   \begin{tikzpicture}
      \begin{axis} [
            width = \columnwidth,
            height = 0.7\columnwidth,
            enlarge x limits = {upper, value=0.05},
            xtick distance = 4,
            major x tick style = transparent,
            ymin = 0,
            ymax = 1000,
            yticklabel style = {rotate=90},
            ylabel={Operations/Second (Thousands)},
            xlabel={Number of Concurrent Threads},
            %label style = {font=\small},
            xmin = 0,
            xmax = 32,
            legend style={at={(0.02, 0.98)}, anchor=north west}%, font=\small}
         ]
      %\draw[dotted, very thick] (axis cs: 0,669) -- (axis cs: 33,669)
      %   node[pos=.65, above] {relative device limit};
      \addplot [
         style={Plum},
         line width=1.5pt,
         mark=*,
         nodes near coords,
         visualization depends on=\thisrow{Alignment} \as \alignment,
         visualization depends on=\thisrow{YShift} \as \yshift,
         %visualization depends on=\thisrow{Opacity} \as \opacity,
         every node near coord/.append style={anchor=west, rotate=90}
      ]
         table [
            col sep=space,
         ] {data/read-concurrency/splinterdb.csv};
      \addplot [
         style={RoyalBlue},
         line width=1.5pt,
         mark=square*,
         nodes near coords,
         visualization depends on=\thisrow{Alignment} \as \alignment,
         visualization depends on=\thisrow{YShift} \as \yshift,
         visualization depends on=\thisrow{Opacity} \as \opacity,
         every node near coord/.append style={anchor=west, rotate = 330}
      ]
         table [
            col sep=space,
         ] {data/read-concurrency/rocksdb.csv};
      \addplot [
         style={ForestGreen},
         line width=1.5pt,
         mark=triangle*,
         nodes near coords,
         visualization depends on=\thisrow{Alignment} \as \alignment,
         visualization depends on=\thisrow{YShift} \as \yshift,
         visualization depends on=\thisrow{Opacity} \as \opacity,
         every node near coord/.append style={anchor=north west, rotate = 330}
      ]
         table [
            col sep=space,
         ] {data/read-concurrency/pebblesdb.csv};
         \legend{\sysname, RocksDB, PebblesDB};
      \end{axis}
   \end{tikzpicture}
   \caption{Read throughput performance (YCSB workload C) by number of threads.
   Each instance performs 160M reads divided evenly between threads. Higher is
   better.}\label{fig:read-concurrency}
\end{figure}

\begin{figure}
   \begin{tikzpicture}
      \begin{axis} [
            width = \columnwidth,
            height = 0.6\columnwidth,
            xtick distance = 4,
            enlarge x limits = {upper, value=0.05},
            major x tick style = transparent,
            ymin = 0,
            ymax = 2500,
            yticklabel style = {rotate=90},
            ylabel={Operations/Second (Thousands)},
            xlabel={Number of Concurrent Threads},
            %label style = {font=\small, align=center},
            xmin = 0,
            xmax = 32,
            legend style={at={(0.95, 0.5)}, anchor=east}
         ]
      %\draw[dotted, very thick] (axis cs: 0,669) -- (axis cs: 33,669)
      %   node[pos=.65, above] {relative device limit};
      \addplot [
         style={Plum},
         line width=1.5pt,
         mark=*,
         nodes near coords,
         visualization depends on=\thisrow{Alignment} \as \alignment,
         visualization depends on=\thisrow{YShift} \as \yshift,
         every node near coord/.append style={anchor=west,yshift=-1,rotate=-45}
      ]
         table [
            col sep=space,
         ] {data/write-concurrency/splinterdb.csv};
      \addplot [
         style={RoyalBlue},
         line width=1.5pt,
         mark=square*,
         nodes near coords,
         visualization depends on=\thisrow{Alignment} \as \alignment,
         visualization depends on=\thisrow{YShift} \as \yshift,
         every node near coord/.append style={anchor=west,rotate=90}
      ]
         table [
            col sep=space,
         ] {data/write-concurrency/rocksdb.csv};
         \legend{\sysname, RocksDB};
      \addplot [
         style={ForestGreen},
         line width=1.5pt,
         mark=triangle*,
         nodes near coords,
         visualization depends on=\thisrow{Alignment} \as \alignment,
         visualization depends on=\thisrow{YShift} \as \yshift,
         every node near coord/.append style={anchor=south east,rotate=0}
      ]
         table [
            col sep=space,
         ] {data/write-concurrency/pebblesdb.csv};
         \legend{\sysname, RocksDB, PebblesDB};
      \end{axis}
   \end{tikzpicture}
   \caption{Write throughput performance (YCSB Load) by number of threads.
   Each instance performs 673M writes divided evenly between threads.  Higher
   is better.}\label{fig:write-concurrency}
\end{figure}

\subsection{Scan Performance}\label{sec:scan}

\begin{figure}
   \begin{tikzpicture}
      \begin{axis} [
            width = \columnwidth,
            height = 0.6\columnwidth,
            xmode = log,
            %xtick = data,
            major x tick style = transparent,
            /pgf/number format/1000 sep={},
            ylabel near ticks,
            %label style = {font=\small, align=center},
            xmin = 1,
            xmax = 100000,
            enlarge x limits=0.05,
            ymin = 0,
            ymax = 3000,
            yticklabel style = {rotate=90},
            ylabel={Effective Throughput (MiB/sec)},
            xlabel={Scan Length in Number of Key-Value Pairs},
            legend style={at={(0.02, 0.99)}, anchor=north west}%, font=\small},
            %every node near coord/.append style={font=\scriptsize},
         ]
      \draw[dotted, very thick] (axis cs: 0,2600) -- (axis cs: 400000,2600)
         node[pos=.6, above] {device read throughput};
      \addplot [
         style={Plum},
         line width=1.5pt,
         mark=*,
         nodes near coords,
         visualization depends on=\thisrow{Alignment} \as \alignment,
         every node near coord/.append style={anchor=\alignment}
      ]
         table [
            col sep=space,
         ] {data/scans/splinterdb.csv};
      \addplot [
         style={RoyalBlue},
         line width=1.5pt,
         mark=*,
         nodes near coords,
         visualization depends on=\thisrow{Alignment} \as \alignment,
         every node near coord/.append style={anchor=\alignment}
      ]
         table [
            col sep=space,
         ] {data/scans/rocksdb.csv};
      \addplot [
         style={ForestGreen},
         line width=1.5pt,
         mark=*,
         nodes near coords,
         visualization depends on=\thisrow{Alignment} \as \alignment,
         every node near coord/.append style={anchor=\alignment}
      ]
         table [
            col sep=space,
         ] {data/scans/pebblesdb.csv};
         \legend{\sysname, RocksDB, PebblesDB};
      \end{axis}
   \end{tikzpicture}
   \caption{Scan throughput in MiB/sec as a function of scan length. For small
   scans, the start up cost dominates, but as the scans get longer, the
   throughput approaches the device's advertised bandwidth
   (2.6GiB/sec). The x-axis is on a log scale. Higher is
   better.}\label{fig:scans}
\end{figure}

An inherent disadvantage of size-tiering is that short scans must search every
branch along the root-to-leaf path to the starting key. Each of these searches
is likely to incur an IO to the device. As a result, as seen in
\cref{fig:ycsb-100B}, \sysname{} with 124B key-value pairs has scan throughput
on small ranges that is about 85\% that of RocksDB. During that workload,
\sysname performed 2.26 GiB/sec of IO, which is within 96\% of the devices
advertised random read capability (short scans of small key-value pairs are
essentially random reads).

However, once the initial search for the successor to the starting key has
completed, the root-to-leaf path within each relevant branch will be in memory.
Together with prefetching, this allows subsequent keys to be fetched at near
disk bandwidth. Therefore, we expect that scans have a relatively high startup
cost for the search to the starting key, followed by a very low iteration cost
of obtaining subsequent keys.

Thus, when the amount of data requested grows to multiple pages, the
disadvantage begins to dissipate. One way this happens is with larger key-value
pairs: with 1kib values, \sysname is about 16\% faster than RocksDB.

Another way this can happen is with scans of more key-value pairs.  We modify
YCSB workload E to have only fixed-length scans of $N$ key-value pairs, where
$N$ is 1, 10, 100, 1K, 10K or 100K.  We perform runs of 10M scans of length 1,
10 and 100, 1M scans of length 1000, 100K scans of length 10000 and 10K scans
of length 100000. Each run is performed on a dataset of 80GiB (with 24B keys and
100B values) and 4GiB memory.

The result is shown in \cref{fig:scans}. Short scans on \sysname have low
effective bandwidth, and in fact the bandwidth scales close to linearly with
the scan length for scans of up to 100 key-value pairs. This suggests that for
scans of this length, the startup cost dominates the iteration cost, which is
as expected.  As the scan length increases, the effective bandwidth of the
scans approaches the device's advertised sequential read bandwidth, delivering
91\% at scans of 1,000 key-value pairs.  At scans as small as 100 key-value
pairs, \sysname{} returns data at nearly half the bandwidth of the device.

%% \subsection{Insertion Latency}\label{sec:latency}

%% \begin{figure}
%%    \begin{tikzpicture}
%%       \begin{axis} [
%%             width = \columnwidth,
%%             height = 0.8\columnwidth,
%%             restrict y to domain=0.0:0.99,
%%             major x tick style = transparent,
%%             ymin = 0,
%%             ymax = 1,
%%             xlabel={Latency in $\mu$s},
%%             ylabel={Cumulative Density},
%%             ylabel near ticks,
%%          ]
%%       \addplot [
%%          style={Plum},
%%          %line width=1.5pt,
%%       ]
%%          table [
%%             col sep=space,
%%             x expr={\thisrowno{0}/1000.0}
%%          ] {data/insert_latency/insert_latency_cdf.csv};
%%       \end{axis}
%%    \end{tikzpicture}
%%    \caption{CDF of insertion latency for YCSB load workload.}
%%    \label{fig:insert-latency-cdf}
%% \end{figure}

%% \Cref{fig:insert-latency-cdf} shows the CDF of insertion latencies in
%% \sysname.  90\% of insertions complete in under 5 $\mu$secs, and 99\%
%% complete in under 30 $\mu$secs.
