\abstract{
Dictionaries are fundamental data structures that associate values to a set of
keys. They form the foundation of most storage systems, and are key to the
performance of many algorithms.

Dictionaries are well studied from an algorithmic perspective, and many
constructions of optimal dictionaries are known. However, these are rarely used
in practice, and the ubiquitious implementation, the log-structured merge tree,
is theoretically suboptimal.

This work studies a collection of dictionary problems, each of which lies
somewhere between theory and practice. These problems take advantage of the
flow of ideas back and forth betwen them, yielding interesting and surprising
results, both where innovations and ideas in systems have influenced
theoretical data structures, and also where those data structures form the
foundation for new highly performant systems.
}
