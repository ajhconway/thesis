\section{Bundle of Arrays Hashing}\label{sec:boa-boa}

A \defn{Bundle of Arrays Hash Table} (\boa{}) is an external-memory dictionary
based on ST-LSMs. In this section, we describe a simple version which is
optimal in the sense of \cref{thm:ip-lower-bound}, but where the query cost
meets the bound only in expectation, not w.h.p. In \cref{sec:boa-refined}, we
give a version that satisfies \cref{thm:boa-cost}.

As a first step, we show that runs with uniformly distributed, $\Theta(\log
N)$-wise independent fingerprints can be searched more quickly than in an
ST-LSMs.

\begin{lemma}\label[lemma]{lem:interpolation-search}
	Let $A$ be a sorted array of $N$ uniformly distributed $\Theta(\log
	N)$-wise independent keys in the range $[0,K)$, and assume $B=\Omega(\log
	N)$. Then $A$ can be written to external memory using $O(N)$ space and
	$O(N/B)$ IOs so that membership in $A$ can be determined in $O(1)$ IOs with
	high probability.
\end{lemma}

\begin{proof}
  First note that, by \cref{lem:chernoff} and Bonferroni's inequality,
  if $N$ balls are thrown into $\Theta(N/\log{N})$ bins uniformly
  and $\Theta(\log N)$-wise independently, then every bin has
  $\Theta(\log{N})$ balls with high probability.

	Divide the range of keys into $N/B$ uniformly sized buckets; that is,
	bucket $i$ contains keys in the range $[(i-1)KB/N,iKB/N)$. Because the keys
	in $A$ are distributed uniformly, and $B= \Omega(\log N)$, every bucket
	contains $\Theta(B)$ keys with high probability.  Let $F$ be the number of
	items in the fullest bucket, and write the keys in each bucket to disk in
	order using $F$ space for each.  Because $F = \Theta(B)$, this takes the
	desired space and IOs.

	Now, to find a key, compute which bucket it belongs to. A constant number
	of IOs will fetch that bucket, whose address is known because all buckets
	have the same size.
\end{proof}

\begin{corollary}
	If an ST-LSM contains uniformly distributed and $\Theta(\log N)$-wise
	independent fingerprints and has growth factor $\grow$, then a query for
	$K$ can be performed in $O(D_K\grow \log_{\grow} N)$ IOs by writing the
	levels as in \cref{lem:interpolation-search}. The insertion/deletion
	cost is unchanged: $O\left(\frac{1}{B}\left(\log_{\grow}
			N+\log_{\frac{M}{B}} N\right)\right)$ amortized IOs.
\end{corollary} 

While the query performance improves by a factor of $\log{N}$, the ST-LSM
is still off of the optimal tradeoff curve of \cref{thm:ip-lower-bound}. In
particular, queries can be at least exponentially slower than optimal. The
\boa{} uses additional structure in order to reduce this query cost.

\subsection{Routing Filters}\label{sec:boa-routing-filter}
The main result of this section is an auxiliary data structure, the
\defn{routing filter}, that improves the query cost of an ST-LSM by a
factor of $\grow$ by further exploiting the log-wise independence of
fingerprints.  Combining these routing filters with fast interpolation
search will yield the BOA, a hashing data structure that is optimal
for large enough $\grow$. 

The purpose of the routing filter is to indicate probabilistically, at each
level, which run contains the fingerprint we are looking for. Each level will
have its own routing filter, defined as follows. For each level $\ell$, let
$h_\ell$ be some number, to be specified below. Let $P_\ell(K)$ be the prefix
consisting of the first $h_\ell$ characters of $K$. The routing filter $F_\ell$
for level $\ell$ is a $\lambda^{h_\ell}$-character array, where $F_\ell[i] = j$
if the $j$th run $R_{\ell,j}$ contains a fingerprint $K$ such that the
$P_\ell(K)=i$, and no later run $R_{\ell,j'}$ (i.e. with $j'>j$) contains such
a fingerprint.

We also modify each run $R_{\ell, j}$ during the merge so that each
fingerprint-value pair contains a \defn{previous field} of 1 additional
character used to specify the previous run containing a fingerprint with the
same prefix, or $j$, to indicate no such run exists. Thus these fingerprint-value
pairs now form a singly linked list whose fingerprints share the same prefix,
and the routing filter points to the run containing the head.

During a query for a fingerprint $K$, first $F_\ell[P_\ell(K)]$ is checked to
find the latest run containing a fingerprint with a matching prefix. Once that
fingerprint-value pair is found, its previous field indicates the next run
which needs to be checked and so on until all fingerprints with matching prefix
in the level are found. Each fingerprint $K'\neq K$ that matches $K$'s prefix
is a false positive.

Such routing filters induce a space/cost tradeoff. The greater $h_\ell$ is, the
more space the table takes but the less likely it is that many runs will have
false positives. The rest of this section shows that when $h_\ell=\log_\lambda
B + \ell$, in other words, when prefixes grow by a character per level, the BOA
lies on the optimal tradeoff curve of \cref{thm:ip-lower-bound}.

Define $\beta$, the \defn{routing table ratio}, to be the ratio of the number
of buckets in the routing filter to the size of a run. The number of entries in
a run on level $\ell$ is $B\lambda^{\ell-1}$, so $\beta =
\lambda^{h_\ell}/B\lambda^{\ell-1}$. We first analyze the per-level
insertion/deletion cost, and then we compute the expected number of false
positives in order to analyze the overall query cost.

\begin{lemma}\label[lemma]{lem:boa-insertion-cost}
	For a \boa{} with growth factor $\grow$ and routing table ratio $\beta$,
	merging a level incurs $\Theta\left(\frac{1}{B}\left(1 +
	\log_{\frac{M}{B}}\grow + \beta\log_{N}{\grow}\right)\right)$ IOs per
	fingerprint.
\end{lemma}

\begin{proof}
	Merging a level requires merging its runs as well as updating the
	next level's routing filter. Merging $\grow$ sorted arrays takes
	$\Theta\left(\frac{1}{B}\left(1 + \log_{M/B}\grow\right)\right)$ IOs per
	fingerprint.
			
	The routing filter is updated by iterating through it and the new run
	sequentially. For each fingerprint $K$ appearing in the run,
	$F_{\ell+1}[P_{h_{\ell+1}}(K)]$ is copied to the previous field in the run,
	and $F_{\ell+1}[P_{h_{\ell+1}}(K)]$ is set to the number of the current
	run. Each entry in the routing filter is a character, and the routing
	filter has $\beta$ entries for each new fingerprint. Thus, it requires
	$\Theta\left(\frac{\beta}{B}\log_{N}{\grow}\right)$ IOs per fingerprint to
	update sequentially. 
\end{proof}

\begin{lemma}\label[lemma]{lem:boa-false-positive}
	For a \boa{} with growth factor $\grow$ and routing table ratio $\beta$,
	querying a fingerprint $K$ on a given level incurs at most
	$\frac{\lambda}{\beta}$ false positives in expectation.
\end{lemma}

\begin{proof}
	Given some enumeration of the fingerprints in level $\ell$, which are not
	equal to $K$, denote the $i$th such fingerprint by $K_i$. Some of these may
	be duplicates. Let $X_i$ be the indicator random variable, which is 1
	if $P_\ell(K)=P_\ell(K_i)$ and 0 otherwise. $K$ and $K_i$ are
	uniformly distributed and their bits are pairwise independent.  Thus
	$\E{X_i} \leq \frac{1}{\lambda^{h_\ell}}$. The expected number of
	fingerprints (excluding $K$) in the level with prefix $P_\ell(K)$ is thus
	at most $\sum_{i=1}^{B \lambda^\ell}\E{X_i} \leq \frac{B
	\lambda^\ell}{\lambda^{h_\ell}} = \frac{\grow}{\beta}.$
\end{proof}

\begin{lemma}\label[lemma]{lem:boa-cost-beta}
	A \boa{} with growth factor $\grow$ and routing table ratio $\beta$ has
	insertion/deletion cost $O\left(\frac{1}{B}\left(\beta +
	\log_{\frac{M}{B}} N + \log_\lambda N\right)\right)$. A query for
	fingerprint $K$ has expected cost\\
	$O\left(\frac{\lambda}{\beta}D_K\log_\lambda N\right)$, where $D_K$ is the
	duplication count of $K$.
\end{lemma}

\begin{proof}
	Because a \boa{} has $\log_\lambda N$ levels, the insertion cost follows from
	\cref{lem:boa-insertion-cost}.

	To query for a fingerprint $K$, the routing filter on each level is
	checked, which incurs $O(\log_\lambda N)$ IOs. These routing filters return
	a collection of runs which contain up to $D_K$ true positives and an
	expected $O\left(\frac{\lambda}{\beta}\log_\lambda N\right)$ false
	positives, by \cref{lem:boa-false-positive}. By
	\cref{lem:interpolation-search}, each run can be checked in $O(1)$ IOs.
\end{proof}

So for a fixed $\grow$, there is no advantage to choosing $\beta =
\omega(\grow)$. On the other hand, $\beta = o(\grow)$ is suboptimal, because
then choosing $\beta'=\grow'=\beta$ changes a linear factor in the query cost
to a logarithmic one. Therefore, it is optimal to choose $\beta =
\Theta(\grow)$, and in what follows we will fix $\beta = \grow$. Thus,

\begin{lemma}\label{thm:boa-cost-expectation}
	A \boa{} supports $N$ insertions and deletions with amortized
	per entry cost of $O\left(\left(\grow + \log_{\frac{M}{B}} N +
	\log_\grow{N} \right)/B\right)$ IOs, for any $\grow > 1$. A query for a key
	$K$ costs $O(D_K\log_\grow{N})$ IOs in expectation, where $D_K$ is the
	duplication count of $K$.
\end{lemma}

