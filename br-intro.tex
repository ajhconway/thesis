%!TEX root = paper.tex

\section{Introduction}

Balls-and-bins games have been a successful tool for modeling load balancing
problems~\cite{DBLP:journals/tpds/Mitzenmacher01,DBLP:conf/stoc/AdlerCMR95,DBLP:conf/esa/AdlerBS98,DBLP:journals/siamcomp/AzarBKU99,DBLP:conf/random/ColeFMMRSU98,DBLP:conf/stoc/ColeMHMRSSV98,DBLP:journals/rsa/CzumajS01,DBLP:conf/focs/Mitzenmacher96,DBLP:journals/mst/Mitzenmacher99,DBLP:journals/jacm/Vocking03,DBLP:journals/rsa/BringmannSSS16,DBLP:journals/algorithmica/Park17,WestZaWa16,DBLP:journals/talg/Farach-ColtonFM09,DBLP:conf/icalp/ConwayFS18}.
For example, they can be used to study the average and worst-case occupancy of
buckets in a hash table~\cite{DBLP:journals/corr/abs-0901-1155}, the
worst-case load on nodes in a distributed
cluster~\cite{PetrovaOlMa10,DBLP:conf/podc/BerenbrinkFKMNW16} and even the amount
of time customers wait in line at the grocery store~\cite{DBLP:journals/mst/Mitzenmacher99}.  In
all these load-balancing problems, balls-and-bins games are used to study how
to distribute load evenly across the resource being allocated.

In this paper we study a new scenario, which we refer to as the
\defn{ball-recycling game}, defined as follows:
\begin{displayquote}
	Throw $m$ balls into $n$ bins i.i.d.\ according to a given probability
	distribution $\p$.  Then, at each time step, pick a non-empty bin and
	\defn{recycle} its balls: take the balls from the selected bin and re-throw
	them according to $\p$.
\end{displayquote}
We call a bin-picking method a \defn{recycling strategy} and define its
\defn{recycling rate} to be the expected number of balls recycled in the
stationary distribution (when it exists).

The ball-recycling game models \defn{insertion buffers} and \defn{update
buffers}, which are widely used to speed up insertions in databases by batching
updates to blocks on disk. The recycling rate of a recycling strategy
corresponds to the speed-up obtained by an insertion buffer, so the goal
studied in this paper is how to maximize the recycling rate. This relationship
is described in \cref{sec:br-motivation}, and the experiments in
\cref{sec:br-experiments} demonstrate that it holds in practice.

In this paper, we present results for ball recycling for both general $\p$ and
for the special case of uniform $\p$, which we denote by $\uni$.  As we explain
in \cref{sec:br-motivation}, these distributions correspond to update and
insertion buffers, respectively.

We focus on three natural recycling strategies:
\begin{itemize}
\item \FB: A greedy strategy that recycles the bin with the most balls.
\item \RB: A strategy that picks a ball uniformly at random and recycles its
	bin.
\item \GG: A strategy that picks the bins in round-robin fashion; after a bin
	is picked, the next bin picked is its non-empty successor.
\end{itemize}
Let $\halfnormp = (\sum \sqrt{\p_i})^2$  be the half quasi-norm of $\p$.  We
achieve the following result for general $\p$.

\begin{theorem}[\cref{sec:br-nonuniform}]\label{thm:random-opt}
	Consider a ball-recycling game with $m$ balls and $n$ bins, where the balls
	are distributed into the bins i.i.d.\ according to distribution $\p$. Then
	\RB{} is $\Theta(1)$-optimal.

        It achieves recycling rate
	$\mathcal{R}^\textrm{RB}$:
	\begin{enumerate}
		\item If $m \geq n$,
			\[\mathcal{R}^\textrm{RB} = \Theta\left(\frac{m}{\halfnormp}\right).\]
		\item If $m < n$, let $L$ be the $m$ lowest-weight bins, $q = \sum_{\ell\in
			L} p_\ell$, and $\mathcal{R}_L^\textrm{RB}$ be the recycle rate of
			\RB restricted to $L$. Then,
			\[\mathcal{R}^\textrm{RB} =
			\Theta\left(\min\left(\mathcal{R}_L^\textrm{RB}, 1/q\right)\right).\]
	\end{enumerate}
\end{theorem}

In order to establish this result, we first show that no recycling strategy can
achieve a higher recycling rate than $(2m+n)/\halfnormp$.  This directly
establishes optimality when $m = \Omega(n)$. For $m=o(n)$, we show that \RB
performs as well as another strategy, \AE, which takes an optimal strategy on a
subset of high-weight bins and turns it into an optimal strategy on all the
bins.

Interestingly, the greedy strategy \FB is not generally optimal, and in
particular:
\begin{observation}
There are distributions for which \FB is pessimal, that is, it
recycles at most 2 balls per round whereas OPT recycles almost $m$
balls per round.
\end{observation}
For example, consider the \defn{skyscraper distribution}, where $\p_0 =
1-1/n+1/n^2$ and $\p_i = 1/n^2$, for $0<i\leq n-1$.  Suppose that $m=\sqrt{n}$.
Then \FB will pick bin $0$ every time until it has at most one ball, at which
point it will pick another bin, which will almost certainly have $1$ ball in
it.  Thus, the recycling rate of \FB drops below $2$.  Suppose, instead, that
we recycle the least-full non-empty bin.  In this case, every approximately
$\sqrt{n}$ rounds, a ball lands in a low-probability bin and is promptly
returned to bin $0$. Thus, the recycling rate of this strategy is nearly $m$.
Thus, \FB is pessimal for this distribution.

However, the uniform distribution is of particular importance to insertion
buffers for databases based on \btrees{}. This is because (arbitrary) random
\btree{} insertions are nearly uniformly distributed across the leaves of the
\btree{}, as we show in \cref{sec:br-motivation}. On the uniform distribution, \FB
and \GG are optimal even up to lower order terms:

\begin{theorem}[\cref{sec:br-uniform}]\label{thm:fullestbin}
  \FB and \GG are optimal to within an additive constant for the
  ball-recycling game with distribution $\uni$ 
  for any $n$ and $m$.  They each achieve a
  recycling rate of at least $2m/(n+1)$, whereas no recycling strategy can
  achieve a recycling rate greater than $2m/n + 1$.
\end{theorem}

In this case, \RB is only optimal to within a multiplicative constant
in the following range:

\begin{theorem}[\cref{sec:br-uniform}]\label{thm:rbuniform}
	On the uniform distribution $\uni$, for sufficiently large
        $m$, \RB is at least $(1/2+1/(2^3
	3^4))$-optimal and at most $(1-3/1000)$-optimal.
\end{theorem}

Thus, we establish some surprising results: that \FB can perform poorly for
arbitrary $\p$ but is optimal for $\uni$, up to lower-order terms; and that \RB
is asymptotically optimal for any $\p$ and in particular is more than
$1/2$-optimal but not quite optimal for $\uni$.

In \cref{sec:br-experiments}, we present experimental results showing that our
analytical results for the ball recycling problem closely match performance
results in real databases.  We describe the recycling strategies of several
commercial and open-source database systems. In particular we focus on InnoDB,
a \btree{} that uses a variant of \RB.

Our results suggest that \FB or \GG would be a better choice than \RB for
InnoDB. In particular, \GG requires almost no additional bookkeeping, and can
be implemented in InnoDB with a change of only a few lines of code.
With this implementation, we measured a 30\% improvement in its
insertion-buffer flushing rate, which is in line with our theoretical results.

We conclude that ball recycling is a natural hitherto unexplored balls-and-bins
game that closely models a widely deployed method for improving the performance
of databases. Moreover, this is the first application (to our knowledge) of a
balls-and-bins game to the throughput of a system. This is in contrast to past
balls-and-bins analyses, which modeled load balancing and latency.


%%% Local Variables:
%%% mode: latex
%%% TeX-master: "paper.tex"
%%% End:
